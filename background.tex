%!TEX root = etdrtemplate.tex
% +--------------------------------------------------------------------+
% | Sample Chapter 2
% +--------------------------------------------------------------------+

\cleardoublepage

% +--------------------------------------------------------------------+
% | Replace "This is Chapter 2" below with the title of your chapter.
% | LaTeX will automatically number the chapters.
% +--------------------------------------------------------------------+

\chapter{Background}
\label{background}

This chapter is brief introduction of all technologies and tools used in this thesis. It is SPARK/Ada programming language and its tools (GNAT Programming Studio, Sireum Bakar, GNATprove), AADL modeling language, BLESS (AADL annex language). There is also overview of the context in which this work has been made: Integrated Clinical Environment standard (ICE) and PCA Pump (ICE compliant device). This is followed by main topic of the thesis: code generation from AADL and analysis of existing AADL translators (Ocarina, RAMSES).



\section{Integrated Clinical Environment}
\label{background:ice}
Medical devices are safety-critical systems.
%http://santos.cis.ksu.edu/MDCF/doc/ICE-Motivation.pdf
%http://santos.cis.ksu.edu/MDCF/doc/MDCF-Tutorial-Overview.pdf
Medical Devices Coordination Framework is an open, experimental ICE-compliant platform to bring together academic researchers, industry vendors, and government regulators.
Medical Devices, which are ICE compliant can be connected to MDCF. It enables Medical Devices cooperation.
[add some pictures etc.]


\section{AADL}
\label{background:aadl}

AADL stands for Architecture Analysis \& Design Language. The aim of the AADL is to allow the description of Distributed Real-Time Embedded (DRE) systems by assembling separately developed blocks. Thus it focuses on the definition of clear block interfaces, and separates the implementations from those interfaces. AADL allows for the description of both software and hardware parts of a system \footnote{http://penelope.enst.fr/aadl}. 

AADL has its roots in DARPA \footnote{http://www.darpa.mil} funded research. The first verision (1.0) was approved in 2004 under technical leadership of Peter Feiler \footnote{http://wiki.sei.cmu.edu/aadl/index.php/The\_Story\_of\_AADL/}. AADL is develop by SAE AADL committee \footnote{https://wiki.sei.cmu.edu/aadl/index.php/Main\_Page}. AADL version 2.0 was published in January 2009. The most recent version (2.1) was published in September 2012 \footnote{https://wiki.sei.cmu.edu/aadl/index.php/Standardization}.

AADL is a language for Model-Based Engineering \cite{AadlBook}. It can be represented in textual and graphical form. There are tools (like Osate \ref{background:aadl:osate}), which transforms textual representation into graphical. There is also possiblity to represent AADL in XML (using 3rd party tools). An example AADL model called Thermometer is shown in graphical representation in figure \ref{figure:patient_thermometer} and in textual representation in listing \ref{listing:patient_thermometer}.

\begin{figure}[ht]%t=top, b=bottom, h=here
    \begin{center}
    	\includegraphics[height=1.5in]{figures/patient_thermometer.png}
    	\caption{AADL model of simple thermometer}
    \end{center}
    \label{figure:patient_thermometer}
\end{figure}

\begin{lstlisting}[language=aadl, frame=single, gobble=0, caption={AADL model of simple thermometer}, label={listing:patient_thermometer}]
	package Thermometer
	public
	with Base_Types;
		system patient_thermometer
		end patient_thermometer;

		system implementation patient_thermometer.impl
		subcomponents
			thermomether : device thermometer_device.impl;
			opi : device operator_interface.impl;
		connections
			tdn : port thermomether.temp -> opi.display;
		end patient_thermometer.impl;

		device operator_interface
		features
			display : in data port Base_Types::Integer;
		end operator_interface;

		device implementation operator_interface.impl
		end operator_interface.impl;

		device thermometer_device
		features
			temp : out data port Base_Types::Integer;
		end thermometer_device;

		device implementation thermometer_device.impl
		end thermometer_device.impl;
	end Thermometer;
\end{lstlisting} 

%another example: https://wiki.sei.cmu.edu/aadl/images/7/73/AADLV2Overview-AADLUserDay-Feb_2010.pdf (slide 16)

Recently AADL becomes a new market standard. There are lots of tools for AADL models analysis, such as: STOOD \footnote{http://www.ellidiss.com/products/stood}, ADELE \footnote{https://wiki.sei.cmu.edu/aadl/index.php/Adele}, Cheddar \footnote{http://beru.univ-brest.fr/~singhoff/cheddar}, AADLInspector \footnote{http://www.ellidiss.com/products/aadl-inspector} or Ocarina \footnote{http://www.openaadl.org}.

What is important, AADL is for architectural description. It should not be compared with UML suites, which allows to link with source code.


\subsection{OSATE}
\label{background:aadl:osate}
Open Source AADL Tool Environment (OSATE) is a set of plug-ins on top of the open-source Eclipse platform. It provides a toolset for front-end processing of AADL models. OSATE is developed mainly by SEI (Software Engineering Institute - CMU) \footnote{http://www.aadl.info/aadl/currentsite/tool/osate.html}. Latest available version of OSATE in the time when this work was published is OSATE2 \footnote{https://wiki.sei.cmu.edu/aadl/index.php/Osate\_2}.



\section{BLESS}
\label{background:bless}
BLESS (Behavior Language for Embedded Systems with Software) is AADL annex sublanguage defining behavior of components. The goal of BLESS is automatically-checked correctness proofs of AADL models of embedded electronic systems with software.

BLESS contains three AADL annex sublanguages:
\begin{itemize} \itemsep1pt \parskip0pt \parsep0pt
	\item Assertion - it can be attached individually to AADL features (e.g. ports)
	\item subBLESS - can be attached only to subprograms; it has only value transformations and Assertions without time expressions
	\item BLESS - it can be attached to AADL thread, device or system components; it contains states, transitions, timeouts, actions, events and Assertions with time expressions...
\end{itemize}

How it fits into the picture. Why it was developed. Corectness prove in AADL + behavior \cite{Bless:Paper}, from which we can generate SPARK/Ada code.



\section{SPARK/Ada}
\label{background:spark}

First version of Ada programming language, Ada 83 was designed to meet the US Department of Defence Requirements formalized in "Steelman" document \footnote{http://www.adahome.com/History/Steelman/steelman.htm}. Since that time, Ada evolved. There were Ada 95, Ada 2005 and now we have Ada 2012 (released in December 10, 2012) \footnote{http://www.ada2012.org}. Ada is activelly used in many Real-World projects\footnote{http://www.seas.gwu.edu/~mfeldman/ada-project-summary.html}, e.g. Aviation (Boeing \footnote{http://archive.adaic.com/projects/atwork/boeing.html}), Railway Transportation, Commercial Rockets, Satellites and even Banking. One of the main goals of Ada is to ensure software corectness and safety.

\begin{figure}[ht]%t=top, b=bottom, h=here
    \begin{center}
    	\includegraphics[height=3.5in]{figures/developer_responsibility_in_ada.png}
    	\caption{Developer responsibility in Ada\protect\footnotemark. }    	
    \end{center}
\end{figure}
\footnotetext{http://www.slideshare.net/AdaCore/ada-2012}

SPARK is a programming language and static verification technology designed specifically for the development of high integrity software. First version was designed over 20 years ago. SPARK has established a track record of use in embedded and critical systems across a diverse range of industrial domains where safety and security are paramount \cite{Barnes:Book}. 

SPARK provides a significant degree of automation in proving exception freedom \cite{Spark:Article}. SPARK excludes some Ada constructs to make static analysis feasible \cite{Spark:Article}. Additionally SPARK contains toolset for Software Verification:
\begin{itemize} \itemsep1pt \parskip0pt \parsep0pt
	\item Examiner - analyse code and ensures that it conforms to the SPARK language; also verify program to some extent using Verification Conditions (VC)
	\item Simplifier - simplify Verification Conditions generated by Examiner
	\item Proof Checker - prove the Verification Conditions
\end{itemize}

First version of SPARK was based on Ada 83. SPARK 2005 is based on Ada 2005. It includes annotation language that support flow analysis and formal verification. Annotations are encoded in Ada comments (via the prefix \lstinline{--#}). It makes every SPARK 2005 programm, valid Ada 2005 program. Figure \ref{listing:Odometer2005} shows example SPARK 2005 package specification.

\begin{lstlisting}[language=ada, frame=single, gobble=0, caption={SPARK 2005 code: Odometer \cite{Barnes:Book}}, label={listing:Odometer2005}]
	package Odometer
	--# own Trip, Total : Integer;
	is
		procedure Zero_Trip;
		--# global out Trip;
		--# derives Trip from ;
		--# post Trip = 0;

		function Read_Trip return Integer;
		--# global in Trip;

		function Read_Total return Integer;
		--# global in Total;

		procedure Inc;
		--# global in out Trip, Total;
		--# derives Trip from Trip & Total from Total;
		--# post Trip = Trip~ + 1 and Total = Total~ + 1;

	end Odometer;
\end{lstlisting} 

SPARK 2005 is a subset of Ada 2005 with annotations. It does not include constructs such as pointers, dynamic memory allocation or recursion \cite{Spark:Article}.

SPARK 2014 \footnote{http://www.spark-2014.org} is based on Ada 2012 programming language targeted at safety- and security-critical applications \cite{Spark2014:Paper}. Since Ada 2012 contains contracts, there is no need to use annotations like in SPARK 2005. Thus SPARK 2014 is subset of Ada 2012. It contains all features of Ada 2012 except:
\begin{itemize} \itemsep1pt \parskip0pt \parsep0pt
 	\item Access types (pointers)
 	\item Exceptions
	\item Aliasing between variables
	\item Concurrency features of Ada (Tasking)
	\item Side effects in expressions and functions
\end{itemize}

SPARK 2014 does not contains Examiner. Instead, proofs are made by gnatPROVE.
The notion of executable contracts in Ada 2012, was inspired by SPARK. Previous Odometer example in SPARK 2014 is shown in figure \ref{listing:Odometer2014}.

\begin{lstlisting}[language=ada2012, frame=single, gobble=0, caption={SPARK 2014 code: Odometer}, label={listing:Odometer2014}]
	package Odometer
	with SPARK_Mode
	Abstract_State => (Trip, Total)
	is
		procedure Zero_Trip
		with Global => (Output => (Trip)),
		   Depends => (Trip => null),
		   Post => (Trip = 0);

		function Read_Trip return Integer
		with Global => (Input => (Trip));

		function Read_Total return Integer
		with Global => (Input => (Total));

		procedure Inc	   
		with Global => (In_Out => (Trip, Total)),
			Depends => (Trip => Trip, Total => Total),
			Post => Trip = Trip'Old + 1 and Total = Total'Old + 1;

	end Odometer;
\end{lstlisting} 

It is possible to mix SPARK 2014 with Ada 2012. However, only the part which is SPARK 2014 compliant will be verified.
% http://docs.adacore.com/spark2014-docs/html/ug/spark_2014.html#mixing-spark-code-and-ada-code

The most popular IDE for SPARK/Ada is GNAT Programming Studio \footnote{http://libre.adacore.com/tools/gps}.

There is also plugin for Eclipse: GNATbench \footnote{https://www.adacore.com/gnatpro/toolsuite/gnatbench/} created by AdaCore. 
Tools for corectness proving.

\subsection{GNAT Programming Studio}
\label{background:spark:gps}
IDE for SPARK/Ada programs development. Includes proving tools. E.g. Sireum Bakar (developed by SAnToS lab) or GNATprove.


\subsection{Sireum Bakar}
\label{background:spark:sireum}
Overview: symbolic execution, Pilar, Kiasan and Alir \cite{Hari:Thesis}.
Sireum Kiasan \cite{Kiasan:Paper} is a tool, which use symbolic execution for finding possible paths in program.
Plugin for GNAT Programming Studio.
Plugin for Eclipse (but only SPARK 2005).


\subsection{GNAT Prove}
\label{background:spark:gnatprove}
GNATprove \footnote{http://www.open-do.org/projects/hi-lite/gnatprove/} is a formal verification tool for SPARK/Ada programs. It interprets SPARK/Ada annotations exactly like they are interpreted at run time during tests.
% http://docs.adacore.com/spark2014-docs/html/ug/gnatprove.html


\subsection{AUnit(remove it?)}
\label{background:spark:aunit}
Overview
AUnit tutorials \cite{AUnitTutorials:Online}
AUnit Cookbook \cite{AUnitCookbook:Online}



\section{PCA Pump}
\label{background:pcapump}
%http://www.santoslab.org/pub/paper/LarsonEtAl13-PCA-Requirements-SEHC-preprint.pdf



\section{AADL/BLESS to SPARK/Ada code generation}
\label{background:codegen}
The ultimate goal of long term research, this thesis is part of is AADL (with BLESS) to SPARK/Ada translation.


\subsection{Ocarina}
\label{background:codegen:ocarina}
Ocarina \cite{Ocarina:Paper,Ocarina:Paper} generates code from an AADL architecture model to an Ada application running on top of PolyORB framework. In this context, PolyORB acts as both the distribution middleware and execution runtime on all targets supported by PolyORB.
It generate Ada 2005 and C code.
Since mid-2009, Telecom ParisTech is no longer involved in Ocarina, and is developping another AADL toolchain, based on Eclipse, codenamed RAMSES \cite{Ocarina:About:Online}.


\subsection{Ramses}
\label{background:codegen:ramses}
RAMSES is a model transformation framework dedicated to the refinement of AADL models.
% http://www.aadl.info/aadl/downloads/committee/feb2013/presentations/RAMSES_status_2013_06_02_format.pdf
% https://wiki.sei.cmu.edu/aadl/index.php/OSATE_2_on_the_command-line