%!TEX root = etdrtemplate.tex

\cleardoublepage


\chapter{Background}
\label{background}

This chapter is brief introduction of all technologies and tools used in this thesis. There are: SPARK Ada programming language and its tools (GNAT Programming Studio, Sireum Bakar, GNATprove), AADL modeling language, BLESS (AADL annex language). There is also overview of the context in which this work has been made: Integrated Clinical Environment standard (ICE) and PCA Pump (ICE compliant device). This is followed by main topic of the thesis: code generation from AADL and analysis of existing AADL translators (Ocarina, RAMSES).



\section{Integrated Clinical Environment}
\label{background:ice}
Medical devices are safety-critical systems.
%http://santos.cis.ksu.edu/MDCF/doc/ICE-Motivation.pdf
%http://santos.cis.ksu.edu/MDCF/doc/MDCF-Tutorial-Overview.pdf
%Shiwei's paper
Medical Devices Coordination Framework is an open, experimental ICE-compliant platform to bring together academic researchers, industry vendors, and government regulators.
Medical Devices, which are ICE compliant can be connected to MDCF. It enables Medical Devices cooperation.
[add some pictures etc.]



\section{AADL}
\label{background:aadl}

AADL stands for Architecture Analysis \& Design Language. The aim of the AADL is to allow the description of Distributed Real-Time Embedded (DRE) systems by assembling separately developed blocks. Thus it focuses on the definition of clear block interfaces, and separates the implementations from those interfaces. AADL allows for the description of both software and hardware parts of a system \footnote{http://penelope.enst.fr/aadl}. 

AADL has its roots in DARPA \footnote{http://www.darpa.mil} funded research. The first version (1.0) was approved in 2004 under technical leadership of Peter Feiler \footnote{http://wiki.sei.cmu.edu/aadl/index.php/The\_Story\_of\_AADL/}. AADL is develop by SAE AADL committee \footnote{https://wiki.sei.cmu.edu/aadl/index.php/Main\_Page}. AADL version 2.0 was published in January 2009. The most recent version (2.1) was published in September 2012 \footnote{https://wiki.sei.cmu.edu/aadl/index.php/Standardization}.

AADL is a language for Model-Based Engineering \cite{AadlBook}. It can be represented in textual and graphical form. There are tools (like Osate \ref{background:aadl:osate}), which transforms textual representation into graphical. There is also possiblity to represent AADL in XML (using 3rd party tools). 

%https://wiki.sei.cmu.edu/aadl/images/7/73/AADLV2Overview-AADLUserDay-Feb_2010.pdf PAGE 24

An example AADL model called Thermometer is shown in graphical representation in figure \ref{figure:patient_thermometer} and in textual representation in listing \ref{listing:patient_thermometer}.

\begin{figure}[ht]%t=top, b=bottom, h=here
    \begin{center}
    	\includegraphics[height=1.5in]{figures/patient_thermometer.png}
    	\caption{AADL model of simple thermometer}
    \end{center}
    \label{figure:patient_thermometer}
\end{figure}

\singlespacing
\begin{lstlisting}[language=aadl, frame=single, gobble=0, caption={AADL model of simple thermometer}, label={listing:patient_thermometer}]
	package Thermometer
	public
	with Base_Types;
		system patient_thermometer
		end patient_thermometer;

		system implementation patient_thermometer.impl
		subcomponents
			thermomether : device thermometer_device.impl;
			opi : device operator_interface.impl;
		connections
			tdn : port thermomether.temp -> opi.display;
		end patient_thermometer.impl;

		device operator_interface
		features
			display : in data port Base_Types::Integer;
		end operator_interface;

		device implementation operator_interface.impl
		end operator_interface.impl;

		device thermometer_device
		features
			temp : out data port Base_Types::Integer;
		end thermometer_device;

		device implementation thermometer_device.impl
		end thermometer_device.impl;
	end Thermometer;
\end{lstlisting} 
\doublespacing
%another example: https://wiki.sei.cmu.edu/aadl/images/7/73/AADLV2Overview-AADLUserDay-Feb_2010.pdf (slide 16)

Recently AADL becomes a new market standard. There are lots of tools for AADL models analysis, such as: STOOD \footnote{http://www.ellidiss.com/products/stood}, ADELE \footnote{https://wiki.sei.cmu.edu/aadl/index.php/Adele}, Cheddar \footnote{http://beru.univ-brest.fr/~singhoff/cheddar}, AADLInspector \footnote{http://www.ellidiss.com/products/aadl-inspector} or Ocarina \footnote{http://www.openaadl.org}.

What is important, AADL is for architectural description. It should not be compared with UML suites, which allows to link with source code.

[poprawic]
%http://en.wikipedia.org/wiki/Architecture_Analysis_%26_Design_Language
The language can be extended with the following methods:

user-defined properties: user can extend the set of applicable properties and add their own to specify their own requirements
language annexes: the core language is enhanced by annex languages that enrich the architecture description. For now, the following annexes have been defined.
Behavior annex: add components behavior with state machines
Error-model annex: specifies fault and propagation concerns
ARINC653 annex: defines modelling patterns for modelling avionics system
Data-Model annex: describes the modelling of specific data constraint with AADL


\subsection{OSATE}
\label{background:aadl:osate}
Open Source AADL Tool Environment (OSATE) is a set of plug-ins on top of the open-source Eclipse platform. It provides a toolset for front-end processing of AADL models. OSATE is developed mainly by SEI (Software Engineering Institute - CMU) \footnote{http://www.aadl.info/aadl/currentsite/tool/osate.html}. Latest available version of OSATE in the time when this work was published is OSATE2 \footnote{https://wiki.sei.cmu.edu/aadl/index.php/Osate\_2}.



\section{BLESS}
\label{background:bless}
BLESS (Behavior Language for Embedded Systems with Software) is AADL annex sublanguage defining behavior of components. The goal of BLESS is automatically-checked correctness proofs of AADL models of embedded electronic systems with software.

BLESS contains three AADL annex sublanguages:
\begin{itemize} \itemsep1pt \parskip0pt \parsep0pt
	\item Assertion - it can be attached individually to AADL features (e.g. ports)
	\item subBLESS - can be attached only to subprograms; it has only value transformations and Assertions without time expressions
	\item BLESS - it can be attached to AADL thread, device or system components; it contains states, transitions, timeouts, actions, events and Assertions with time expressions...
\end{itemize}

How it fits into the picture. Why it was developed. Corectness prove in AADL + behavior \cite{Bless:Paper}, from which we can generate SPARK Ada code.



\section{SPARK Ada}
\label{background:spark}

%http://www.cs.swan.ac.uk/~csetzer/lectures/critsys/09/critsysfinal2.pdf

First version of Ada programming language - Ada 83 - was designed to meet the US Department of Defence Requirements formalized in "Steelman" document \footnote{http://www.adahome.com/History/Steelman/steelman.htm}. Since that time, Ada evolved. There were Ada 95, Ada 2005 and Ada 2012 (released in December 10, 2012) \footnote{http://www.ada2012.org}. Ada is actively used in many Real-World projects\footnote{http://www.seas.gwu.edu/~mfeldman/ada-project-summary.html}, e.g. Aviation (Boeing \footnote{http://archive.adaic.com/projects/atwork/boeing.html}), Railway Transportation, Commercial Rockets, Satellites and even Banking. One of the main goals of Ada is to ensure software correctness and safety.

\begin{figure}[ht]%t=top, b=bottom, h=here
    \begin{center}
    	\includegraphics[height=3.5in]{figures/developer_responsibility_in_ada.png}
    	\caption{Developer responsibility in Ada\protect\footnotemark. }    	
    \end{center}
\end{figure}
\footnotetext{http://www.slideshare.net/AdaCore/ada-2012}

SPARK is a programming language and static verification technology designed specifically for the development of high integrity software. It is a "safe" subset of Ada designed to be susceptible to formal methods, accompanied with a set of approaches and tools. Using SPARK, a developer takes a Z specification and performs a stepwise refinement from the specification to SPARK code. For each refinement step a tool is used to produce verification conditions (VC's), which are mathematical theorems. If the VC's can be proved then the refinement step will be known to be valid. However if the VC's cannot be proved then the refinement step may be erroneous \footnote{http://www.dwheeler.com/lovelace/s17s4.htm}.

% Barnes' 15.7 where SPARK is used

First version was designed over 20 years ago. SPARK has established a track record of use in embedded and critical systems across a diverse range of industrial domains where safety and security are paramount \cite{Barnes:Book}. 

SPARK provides a significant degree of automation in proving exception freedom \cite{Spark:Article}. SPARK excludes some Ada constructs to make static analysis feasible \cite{Spark:Article}. Additionally SPARK contains tool-set for Software Verification:
\begin{itemize} \itemsep1pt \parskip0pt \parsep0pt
	\item Examiner - analyze code and ensures that it conforms to the SPARK language; also verify program to some extent using Verification Conditions (VC)
	\item Simplifier - simplify Verification Conditions generated by Examiner
	\item Proof Checker - prove the Verification Conditions
\end{itemize}

First version of SPARK was based on Ada 83. The second version (SPARK 95) - on Ada 95. SPARK 2005 is based on Ada 2005. It is a subset of Ada 2005 with annotations. The annotation language support flow analysis and formal verification. Annotations are encoded in Ada comments (via the prefix \lstinline{--#}). It makes every SPARK 2005 program, valid Ada 2005 program. Figure \ref{listing:Odometer2005} shows example SPARK 2005 package specification.

\singlespacing
\begin{lstlisting}[language=ada, frame=single, gobble=0, caption={SPARK 2005 code: Odometer \cite{Barnes:Book}}, label={listing:Odometer2005}]
	package Odometer
	--# own Trip, Total : Integer;
	is
		procedure Zero_Trip;
		--# global out Trip;
		--# derives Trip from ;
		--# post Trip = 0;

		function Read_Trip return Integer;
		--# global in Trip;

		function Read_Total return Integer;
		--# global in Total;

		procedure Inc;
		--# global in out Trip, Total;
		--# derives Trip from Trip & Total from Total;
		--# post Trip = Trip~ + 1 and Total = Total~ + 1;

	end Odometer;
\end{lstlisting} 
\doublespacing

SPARK 2005 does not include constructs such as pointers, dynamic memory allocation or recursion \cite{Spark:Article}.

SPARK 2014 \footnote{http://www.spark-2014.org} is based on Ada 2012 programming language targeted at safety- and security-critical applications \cite{Spark2014:Paper}. Since Ada 2012 contains contracts, there is no need to use annotations like in SPARK 2005. Thus SPARK 2014 is subset of Ada 2012. It contains all features of Ada 2012 except:
\begin{itemize} \itemsep1pt \parskip0pt \parsep0pt
 	\item Access types (pointers)
 	\item Exceptions
	\item Aliasing between variables
	\item Concurrency features of Ada (Tasking) - it's part of SPARK 2014 road-map to include support for tasking in the future, although likely not this year
	\item Side effects in expressions and functions
\end{itemize}

Sample mapping from SPARK 2005 to 2014 is shown on table \ref{table:spark2005and2014mapping}. Complete mapping can be found in SPARK 2014 documentation \footnote{http://docs.adacore.com/spark2014-docs/html/lrm/mapping-spec.html} \cite{Spark2014refManual:Online}.

\singlespacing
\begin{table}[!ht]
	\caption{Sample SPARK 2005 to 2014 mapping.}
	\label{table:spark2005and2014mapping}
	\centering
  	\begin{tabular}{ | p{3in} | p{3in} |}
	  	%\multicolumn{1}{c}{\textbf{AADL/BLESS}} & \textbf{SPARK Ada}\\

		\hline
		\multicolumn{1}{|c|}{\textbf{SPARK 2005}} & \multicolumn{1}{|c|}{\textbf{SPARK 2014}} \\ \hline

		\begin{lstlisting}
			--# global in out X, Y;
		\end{lstlisting} 
		& 
		\begin{lstlisting}[language=ada2012]
			with Global  => (In_Out => (X, Y));
		\end{lstlisting} 

		\\ \hline

		\begin{lstlisting}
			--# derives X from Y &
			--#         Y from X;
		\end{lstlisting} 
		& 
		\begin{lstlisting}[language=ada2012]
			Depends => (X => Y,
			            Y => X);
		\end{lstlisting}

		\\ \hline

		\begin{lstlisting}
			--# pre Y /= 0 and
			--#     X > Integer'First;
		\end{lstlisting} 
		& 
		\begin{lstlisting}[language=ada2012]
			with Pre  => Y /= 0 and 
			             X > Integer'First;
		\end{lstlisting}

		\\ \hline

		\begin{lstlisting}
			--# post X = Y~ and Y = X~;
		\end{lstlisting} 
		& 
		\begin{lstlisting}[language=ada2012]
			with Post => (X = Y'Old and Y = X'Old);
		\end{lstlisting} 

		\\ \hline
	\end{tabular}
\end{table}
\doublespacing

SPARK 2014 does not contains Examiner. Instead, proofs are made by gnatPROVE.
The notion of executable contracts in Ada 2012, was inspired by SPARK. The previous Odometer example in SPARK 2014 is shown in figure \ref{listing:Odometer2014}.

\singlespacing
\begin{lstlisting}[language=ada2012, frame=single, gobble=0, caption={SPARK 2014 code: Odometer}, label={listing:Odometer2014}]
	package Odometer
	with SPARK_Mode
	Abstract_State => (Trip, Total)
	is
		procedure Zero_Trip
		with Global => (Output => (Trip)),
		   Depends => (Trip => null),
		   Post => (Trip = 0);

		function Read_Trip return Integer
		with Global => (Input => (Trip));

		function Read_Total return Integer
		with Global => (Input => (Total));

		procedure Inc	   
		with Global => (In_Out => (Trip, Total)),
			Depends => (Trip => Trip, Total => Total),
			Post => Trip = Trip'Old + 1 and Total = Total'Old + 1;

	end Odometer;
\end{lstlisting} 
\doublespacing

Fundamental SPARK contracts:

\singlespacing
\begin{center}
	\begin{longtable}{| p{1.5in} | p{1.5in} | p{3in} |}
		\caption{Fundamental SPARK annotations}
		\label{table:SparkAnnotations}
		\\
		\hline
		\multicolumn{1}{|c|}{\textbf{SPARK 2005}} & \multicolumn{1}{|c|}{\textbf{SPARK 2014}} & \multicolumn{1}{|c|}{\textbf{Description}} \\ \hline
		\endfirsthead

		\multicolumn{3}{c}%
		{{\bfseries \tablename\ \thetable{} -- continued from previous page}} \\
		\hline 
		\multicolumn{1}{|c|}{\textbf{SPARK 2005}} & \multicolumn{1}{|c|}{\textbf{SPARK 2014}} & \multicolumn{1}{|c|}{\textbf{Description}} \\ \hline
		\endhead

		\hline \multicolumn{3}{|r|}{{Continued on next page}} \\ \hline
		\endfoot

		\hline %\hline
		\endlastfoot

		\begin{lstlisting}
			--# global
		\end{lstlisting} 
		& 
		\begin{lstlisting}[language=ada2012]
			Global
		\end{lstlisting} 
		& 
		list of used global variables within subprogram 

		\\ \hline

		\begin{lstlisting}
			--# derives
		\end{lstlisting} 
		& 
		\begin{lstlisting}[language=ada2012]
			Depends
		\end{lstlisting} 
		& 
		describe dependencies between variables

		\\ \hline

		\begin{lstlisting}
			--# own 
		\end{lstlisting} 
		& 
		\begin{lstlisting}[language=ada2012]
			Abstract_State
		\end{lstlisting} 
		& 
		declare variables defined in package body

		\\ \hline

		\begin{lstlisting}
			--# initializes
		\end{lstlisting} 
		& 
		\begin{lstlisting}[language=ada2012]
			initializes
		\end{lstlisting} 
		& 
		indicates variables, which are initialized

		\\ \hline

		\begin{lstlisting}
			--# inherit
		\end{lstlisting} 
		& 
		not needed
		& 
		allows to access entities of other packages

		\\ \hline

		\begin{lstlisting}
			--# pre
		\end{lstlisting} 
		& 
		\begin{lstlisting}[language=ada2012]
			Pre
		\end{lstlisting} 
		& 
		pre condition

		\\ \hline
		

		\begin{lstlisting}
			--# post
		\end{lstlisting} 
		& 
		\begin{lstlisting}[language=ada2012]
			Post
		\end{lstlisting} 
		& 
		post condition

		\\ \hline
		

		\begin{lstlisting}
			--# assert
		\end{lstlisting} 
		& 
		\begin{lstlisting}[language=ada2012]
			Assert
		\end{lstlisting} 
		& 
		assertion

		\\ \hline
	\end{longtable}
\end{center}
\doublespacing

It is possible to mix SPARK 2014 with Ada 2012. However, only the part which is SPARK 2014 compliant will be verified. Usually SPARK is used in the most critical parts of Software Systems \cite{Spark:IndustrialExp}. It means, that some part is written in e.g. Ada or C++ and the rest in SPARK. The reason of that is the SPARK limitation and lack of necessity to verify some modules.
% http://docs.adacore.com/spark2014-docs/html/ug/spark_2014.html#mixing-spark-code-and-ada-code

The most popular IDE for SPARK Ada is GNAT Programming Studio \footnote{http://libre.adacore.com/tools/gps}.

There is also plugin for Eclipse: GNATbench \footnote{https://www.adacore.com/gnatpro/toolsuite/gnatbench/} created by AdaCore. 
Tools for correctness proving.


\subsection{GNAT compiler and Programming Studio}
\label{background:spark:gps}
GNAT compiler is front end of gcc...
IDE for SPARK Ada programs development. Includes proving tools. E.g. Sireum Bakar (developed by SAnToS lab) or GNATprove.


\subsection{Ravenscar Tasking Subset}
\label{background:spark:ravenscar}

RavenSPARK is subset of the SPARK Ravenscar Profile (which is subset of Ada tasking). The Ravenscar Profile provides a subset of the tasking facilities of Ada95 and Ada 2005 suitable for the construction of high-integrity concurrent programs \cite{Ravenscar:Online}.

The Ravenscar Profile is a subset of the tasking model, restricted to meet the real-time community requirements for determinism, schedulability analysis and memory-boundedness, as well as being suitable for mapping to a small and efficient run-time system that supports task synchronization and communication, and which could be certifiable to the highest integrity levels. The concurrency model promoted by the Ravenscar Profile is consistent with the use of tools that allow the static properties of programs to be verified. Potential verification techniques include information flow analysis, schedulability analysis, execution-order analysis and model checking. These techniques allow analysis of a system to be performed throughout its development life cycle, thus avoiding the common problem of finding only during system integration and testing that the design fails to meet its non-functional requirements. \cite{Ravenscar:Article}

Concurrent programs require the use of different specification and verification techniques from sequential programs. For this reason, tasks, protected units and objects, and synchronization features are currently excluded from SPARK 2014 \footnote{http://docs.adacore.com/spark2014-docs/html/lrm/tasks-and-synchronization.html} \cite{Spark2014refManual:Online}.

To create a task, the task type has to be declared and task variable of this type. Ravenscar does not allow dynamic task creation. Thus, all tasks have to exists for the full lifetime of the program. \cite{IssuesWithRavenscar:Paper} Tasks can be declared only in packages. Not in subprograms or in other tasks. \cite{Barnes:Book} The priority of each tasks has to be specified by \lstinline{pragma Priority}. [what is priorities range?] Listing \ref{lst:SampleTask} shows sample package with two tasks.

\singlespacing
\begin{lstlisting}[frame=single, gobble=0, caption={Sample tasks}, label={lst:SampleTask}]
	package Some_Pkg
	--# own task t1 : Task1;
	--#     task t2 : Task2;
	is
		task type Task1
		is
			pragma Priority(10);
		end Task1;

		task type Task2
		is
			pragma Priority(9);
		end Task2;

	end Some_Pkg;
\end{lstlisting} 
\doublespacing

Declared tasks have to be implemented in the package body (listing \ref{lst:SampleTaskBody}).

\singlespacing
\begin{lstlisting}[frame=single, gobble=0, caption={Sample tasks body}, label={lst:SampleTaskBody}]
	package body Some_Pkg
	is
		t1 : Task1;
		t2 : Task2;

		task body Task1
		is
		begin
			loop
				-- implementation;
			end loop;
		end Task1;

		task body Task2
		is
		begin
			loop
				-- implementation;
			end loop;
		end Task2;

	end Some_Pkg;
\end{lstlisting} 
\doublespacing

There are two ways to access variable in different tasks:
\begin{itemize}
    \item It has to be protected object
    \item It has to be atomic type
\end{itemize}


Protected object encapsulate variable, in such a way that it is accessible, only through protected subprograms. This mechanism use locking, to ensure atomicity. Protected type declaration is similar to task: specification and body has to be defined. Listing \ref{lst:SampleTasksWithProtectedType} shows sample tasks with protected type \lstinline{Integer_Store}, which enable to share Integer variable between tasks. What is important, protected type has to be declared before tasks, which will use it. Otherwise, it will not be visible for them.

\singlespacing
\begin{lstlisting}[frame=single, gobble=0, caption={Sample tasks with protected object}, label={lst:SampleTasksWithProtectedType}]
	package Some_Pkg
	--# own protected Shared_Var : Integer_Store (Priority => 11);
	--#     task t1 : Task1;
	--#     task t2 : Task2;
	is
	    protected type Integer_Store
	    is
	        pragma Priority (11);

	        function Get return Integer;
	        --# global in Integer_Store;

	        procedure Put(X : in Integer);
	        --# global out Integer_Store;
	        --# derives Integer_Store from X;
	    private
	        TheStoredData : Integer := 0;
	    end Integer_Store;

	    task type Task1
	      --# global out Shared_Var;
	    is
	        pragma Priority(10);
	    end Task1;

	    task type Task2
	      --# global in Shared_Var;
	    is
	        pragma Priority(9);
	    end Task2;

	end Some_Pkg;
\end{lstlisting}
\doublespacing

Protected type body also has to be defined in package body (listing \ref{lst:SampleTasksWithProtectedTypeBody}).

\singlespacing
\begin{lstlisting}[frame=single, gobble=0, caption={Sample tasks with protected object body}, label={lst:SampleTasksWithProtectedTypeBody}]
	package body Some_Pkg
	is
	    Shared_Var : Integer_Store;
	    t1 : Task1;
	    t2 : Task2;

	    protected body Integer_Store is
	        function Get return Integer
	        --# global in TheStoredData;
	        is
	        begin
	            return TheStoredData;
	        end Get;

	        procedure Put(X : in Integer)
	        --# global out TheStoredData;
	        --# derives TheStoredData from X;
	        is
	        begin
	            TheStoredData := X;
	        end Put;
	    end Integer_Store;

	    task body Task1
	    is
	    begin
	        loop
	            Shared_Var.Put(5);
	        end loop;
	    end Task1;

	    task body Task2
	    is
	        Local_Var : Integer;
	    begin
	        loop
	            Local_Var := Shared_Var.Get;
	        end loop;
	    end Task2;

	end Some_Pkg;
\end{lstlisting} 
\doublespacing

\lstinline{Task1} is writing to \lstinline{Shared_Var} and \lstinline{Task2} is reading \lstinline{Shared_Var}. The highest priority is assigned to protected object, to ensure atomicity during operations on it. The lowest priority is assigned to \lstinline{Task2}, which is reading \lstinline{Shared_Var}. Reading is usually less expensive operation than writing. Thus, to avoid starvation, \lstinline{Task1} has higher priority than \lstinline{Task2}. Notice, that \lstinline{Shared_Var} is declared in package body, but refined in package specification.

Protected variables may not be used in proof contexts. Thus, if we try to use protected variable in proofs (pre- or postcondition), then SPARK Examiner returns \lstinline{Semantic Error 940 - Variable is a protected own variable. Protected variables may not be used in proof contexts.}. Formal reasoning about interactions and especially temporal properties require other techniques such as model checking and lie outside the scope of SPARK \cite{Barnes:Book}. To preserve opportunity to use pre- and postconditions, atomic types have to be used.

To declare atomic type, we have to use \lstinline{pragma Atomic}. However, there is restriction, that \lstinline{pragma Atomic} cannot be applied to predefined type such as Integer. Thus, we have to define our custom type (which can be just rename of Integer) and apply \lstinline{pragma Atomic} on this type. Listing \ref{lst:SampleTasksWithAtomicType} presents previous example with atomic types instead of protected objects.

\singlespacing
\begin{lstlisting}[frame=single, gobble=0, caption={Sample tasks with atomic type}, label={lst:SampleTasksWithAtomicType}]
	package Some_Pkg
	--# own Shared_Var;
	--#     task t1 : Task1;
	--#     task t2 : Task2;
	--# initializes Shared_Var;
	is
	    type Int32 is new Integer;
	    
	    task type Task1
	      --# global out Shared_Var;
	    is
	        pragma Priority(10);
	    end Task1;

	    task type Task2
	      --# global in Shared_Var;
	    is
	        pragma Priority(9);
	    end Task2;

	end Some_Pkg;

	package body Some_Pkg
	is    
	    Shared_Var : Int32 := 0;
	    t1 : Task1;
	    t2 : Task2;

	    task body Task1
	    is
	    begin
	        loop
	            Shared_Var := 5;
	        end loop;
	    end Task1;

	    task body Task2
	    is
	        Local_Var : Integer;
	    begin
	        loop
	            Local_Var := Integer(Shared_Var);
	        end loop;
	    end Task2;

	end Some_Pkg;
\end{lstlisting}
\doublespacing

Be aware that \lstinline{pragma atomic} does not guaranty atomicity. In most cases, atomic types should not be used for tasking. Instead, protected types should be used.
% precise! http://en.wikibooks.org/wiki/Ada_Programming/Pragmas/Atomic

Another important thing in tasking is Time library: \lstinline{Ada.Real_Time}. It allows to run task periodically, using \lstinline{delay until} statement, which suspends task until specified time. To use \lstinline{delay} in the task, it has to be declared in \lstinline{declare} annotation: \lstinline{--# declare delay;} \cite{Barnes:Book}.

Details about tasking in SPARK are well described in Chapter 8 of Barnes' book \cite{Barnes:Book}. The "Guide for the use of the Ada Ravenscar profile in high integrity systems" \cite{Ravenscar:Article} and the official Ravenscar Profile documentation (which includes examples) \cite{Ravenscar:Online} might be useful as well. The limitations of Tasking in SPARK are reviewed in Audsley's and Welllings' paper \cite{IssuesWithRavenscar:Paper}.


\section{SPARK Ada Verification}
\label{background:sparkverification}

%http://docs.adacore.com/sparkdocs-docs/

Testing vs Verification (form 721 slides): Testing starts with a set of possible test cases, simulates the system on each input, and observes the behavior. In general, testing does not cover all possible executions. On the other hand, verification establishes correctness for all possible execution sequences.
Techniques for Verification:
\begin{itemize}
	\item Formal verification: prove mathematically that the program is correct – this can be difficult for large programs.
	\item Correctness by construction: follow a well- defined methodology for constructing programs.
	\item Model checking: enumerate all possible executions and states, and check each state for correctness.
\end{itemize}

SPARK tools
FDL is the modelling language of the SPARK proof tools. 

\begin{figure}[ht]%t=top, b=bottom, h=here
    \begin{center}
    	\includegraphics[height=7in]{figures/spark-tools.png}
    	\caption{Relationship of the Examiner and Proof Tools\protect\footnotemark.}
    \end{center}
\end{figure}
\footnotetext{http://docs.adacore.com/sparkdocs\-docs/Examiner\_UM.htm}



\subsection{SPARK Examiner}
\label{verification:examiner}

The main SPARK verification tool is Examiner. It supports several levels of analysis:
\begin{itemize}
	\item checking of SPARK language syntactic and static semantic rules
	\item data flow analysis
	\item data and information flow analysis
	\item formal program verification via generation of verification conditions
	\item proof of absence of run-time errors
	\item dead path analysis
\end{itemize}

There is also an option to make the Examiner perform syntax checks only. Using this option on a source file does not require access to any other units on which the file depends, so files can be syntax checked on an individual basis. This allows any syntax errors to be corrected before the file is included in a complex examination.  This option must only be used as a pre-processor: the absence of syntax errors does NOT indicate that the source text is a legal SPARK program. \cite{Examiner:Online} (THIS PART IS COPY AND PASTE FROM Examiner doc - is it ok?)

Put here some examples: method without contract, examine, add specification, pass Examiner.

During implementation, code was regularly checked using SPARK Examiner.

What is very important, Examiner can perform data and information analysis of Ravenscar programs in exactly the same manner as for sequential programs \cite{Ravenscar:Online}. Unfortunately it does not allow protected objects in proof annotations (pre- and post-conditions).

When some parts of the system are written in full Ada (with non-valid SPARK constructs), then Examiner returns error. Ada parts can be excluded from Examiner analysis using \lstinline{--# hide} annotation. The, only warning \lstinline{10 - The body of subprogram Main is hidden - hidden text is ignored by the Examiner.} is returned by Examiner.

Examiner use SPARK index file to locate files necessary for verification. \cite{Barnes:Book}

Examiner can be used with \lstinline{spark} command and appropriate flags described in Examiner Manual \cite{Examiner:Online}.

%http://docs.adacore.com/sparkdocs-docs/SPARK_GPS.htm
%[screenshot - take from 721 paper]
To use Examiner in GNAT Programming Studio:
\begin{itemize}
	\item Run SPARK Make (right click on project / SPARK / SPARK Make)
	\item Set SPARK index file (to spark.idx generated by SPARKMake) [add photo from 721 paper]
	\item (optionally) set configuration file (Standard.ads)
	\item Choose appropriate version of SPARK (95 or 2005)
	\item Choose mode: Sequential (for single tasking programs) or Ravenscar (for multitasking programs)
\end{itemize}

To generate verification conditions (VCs), the \lstinline{-vcg} switch has to be used. It can be set in GNAT Programming Studio (Project / Edit project properties / Switches / Examiner / Generate VCs).
In addition to verification conditions, Examiner can check dead path conjectures. It checks, whether all of the program is useful. To generate dead path conjectures, the \lstinline{-dpc} switch has to be used. It can be also set in GNAT Programming Studio (Project / Edit project properties / Switches / Examiner / Generate DPCs).


\subsubsection{Flow analysis}
\label{verification:examiner:flowanalysis}
%http://www.cs.swan.ac.uk/~csetzer/lectures/critsys/09/critsysfinal2.pdf
There are two types of flow analysis:
\begin{itemize}
	\item Data flow analysis:
	\begin{itemize}
		\item Checks input/output behavior of parameters and variables.
		\item Checks initialization of variables.
		\item Checks that changed and imported variables are used later (possibly as output variables).
	\end{itemize}
	\item Information flow analysis - verifies interdependencies between variables.
\end{itemize}

In data flow analysis, Examiner checks if input parameters are not modified, but used at least once (in at least one branch of program). In the same factor, output parameters cannot be read (before initialization) and has to be initialized (in all branches of program). Input/output parameters has to be both read and write (changed). In similar way, Examiner verify the global variables (specified in annotations). Functions can use only input parameters and can only read global variables. Therefore functions do not have side effects. 

Global variables defined in package body (thus private) has to be declared by \lstinline{--# own} annotation in package specification. If variable is also initialized, \lstinline{--# initializes} annotation has to be used. In Ada, to use package in another package, \lstinline{with} clause has to be used. In SPARK Ada, additionally \lstinline{--# inherits} annotation has to be specified.

In information flow analysis, dependencies between variables are analyzed. These dependencies are specified by \lstinline{--# derives} annotation.


\subsubsection{Verification conditions}
\label{verification:examiner:vc}

To generate verification conditions, two kinds of annotations are relevant for Examiner:
\begin{itemize}
	\item pre-conditions: \lstinline{--# pre}
	\item post-conditions: \lstinline{--# post}
\end{itemize}

Notion of pre- and post-conditions represents Hoare logic. More precisely, Hoare triple: 

\begin{equation} \label{eq:hoare_triple}
	\{P\} C \{Q\}
\end{equation}

P and Q are assertions. C is a command (action) performed between them. P is pre-condition and Q is post-condition.

Additionally, assertions (\lstinline{--# assert}) and checks (\lstinline{--# check}) can be specified in procedure body. Then additional verification conditions are generated.

Functions does not have side effects (as stated in \ref{verification:examiner:flowanalysis}), thus only pre-condition can be applied. However, there is annotation \lstinline{--# return}, which specify function return value.

Verification conditions are generated depended on number of paths in subprogram. Analysis are perform backwards, in other words: we start from post-conditions and consider what must holds before. Flow analysis is well described in chapter 11 of Barnes' book \cite{Barnes:Book}.



\subsection{SPARK Simplifier}
\label{verification:simplifier}

Simplifier can discharge (prove correctness) of verification conditions (VCs) generated by Examiner, but not proved by Examiner. \cite{Simplifier:Online} 



\subsection{ZombieScope}
\label{verification:zombiescope}

ZombieScope is a SPARK tool, that analyses SPARK code to find dead paths, i.e. paths through the code that can never be executed.


\subsection{Victor}
\label{verification:victor}

Victor is a tool to translate SPARK verification conditions (VCs), as generated by the Examiner, into SMT-LIB (file format used to communicate with SMT solvers). \cite{Victor:Online} SMT (Satisfiability Modulo Theories) solver is a tool...
experimental feature
Integrated with SPARKSimp (by -victor flag) and POGS.


\subsection{Proof Checker}
\label{verification:proofchecker}

% Barnes' book: 12.12
Only mention. It is hardcore.

\subsection{SPARKSimp Utility}
\label{verification:sparksimp}
SPARKSimp is a simple "make" style tool for the SPARK analysis tools. Currently, it supports the Simplifier, ZombieScope and ViCToR. It applies the Simplifier (and ViCToR, if requested, please see the Victor\_Wrapper user manual \cite{Victor:Online} for more information) to all .vcg files and ZombieScope to all .dpc files it finds in a directory tree. \cite{SPARKSimp:Online} 



\subsection{Proof Obligation Summarizer (POGS)}
\label{verification:pogs}

The Proof ObliGation Summarizer tool (POGS) reads and understands the structure of the verification condition files. It reports the status of proofs and dead path analyses in a human-readable form. \cite{POGS:Online}

\subsection{AUnit}
\label{background:spark:aunit}
AUnit is Unit Test Framework for Ada language. It can be also applied for verify SPARK Ada programs.
AUnit tutorials \cite{AUnitTutorials:Online}
AUnit Cookbook \cite{AUnitCookbook:Online}


\subsection{Sireum Bakar}
\label{background:spark:sireum}
Overview: symbolic execution, Pilar, Kiasan and Alir \cite{Hari:Thesis}.


\subsubsection{Bakar Kiasan}

Bakar Kiasan \cite{Kiasan:Paper} is a tool, which use symbolic execution for finding possible paths in program.
Plug-in for GNAT Programming Studio (SPARK 2005 and 2014 under development).
Plug-in for Eclipse (only for SPARK 2005).
No support for Ravenscar profile.
Separated sequential parts can be verified (Odometer?). Sequential version of \lstinline{Max_Drug_Per_Hour_Watcher}?



[image: Kiasan report in GPS5 + code highlighted in editor]
Describe Kiasan Report (GUI developed by me in Python PyGTK).
T\# - Test cases (expected behavior)
E\# - Exception cases (unexpected behavior)
coverage
time
etc.

combo to select case
pre- and post state
code highlighting

\subsubsection{Bakar Kiasan}
Quick overview \cite{Hari:Thesis}.


\subsection{GNAT Prove}
\label{background:spark:gnatprove}
GNATprove \footnote{http://www.open-do.org/projects/hi-lite/gnatprove/} is a formal verification tool for SPARK 2014 programs. It interprets SPARK Ada annotations exactly like they are interpreted at run time during tests.
% http://docs.adacore.com/spark2014-docs/html/ug/gnatprove.html
% only for SPARK 2014


\section{AADL/BLESS to SPARK Ada code generation}
\label{background:codegen}
The ultimate goal of long term research, this thesis is part of, is AADL (with BLESS) to SPARK Ada translation.


\subsection{Ocarina}
\label{background:codegen:ocarina}
Ocarina \cite{Ocarina:Paper,Ocarina:Paper} generates code from an AADL architecture model to an Ada application running on top of PolyORB framework. In this context, PolyORB acts as both the distribution middleware and execution runtime on all targets supported by PolyORB.
It generate Ada 2005 and C code.
Since mid-2009, Telecom ParisTech is no longer involved in Ocarina, and is developing another AADL tool-chain, based on Eclipse, codenamed RAMSES \cite{Ocarina:About:Online}.

examples on github

run:
ocarina -x scenario.aadl



\subsection{Ramses}
\label{background:codegen:ramses}
% very shortly
RAMSES is a model transformation framework dedicated to the refinement of AADL models. It contains code generation plug-in.
% http://www.aadl.info/aadl/downloads/committee/feb2013/presentations/RAMSES_status_2013_06_02_format.pdf
% https://wiki.sei.cmu.edu/aadl/index.php/OSATE_2_on_the_command-line