%!TEX root = etdrtemplate.tex
% +--------------------------------------------------------------------+
% | Abstract Page
% +--------------------------------------------------------------------+

\pagestyle{empty}
%\vspace{1cm}
\setlength{\baselineskip}{0.8cm}

%\indent

% +--------------------------------------------------------------------+
% | Enter the text of your abstract below, maximum of 350 words.
% +--------------------------------------------------------------------+


Ada programming language is targeted at embedded and real-time systems.

SPARK/Ada is designed for the development of safety and security critical systems. It contains properties, which allows to prove corectness of program and its entities.


AADL (Architecture Analysis \& Design Language) is modeling language for representing hardware and software. It is used for real-time, safety critical and embedded systems.

BLESS (Behavior Language for Embedded Systems with Software) is AADL annex sublanguage defining behavior of components. The goal of BLESS is automatically-checked correctness proofs of AADL models of embedded electronic systems with software.

Nowadays, we have trend to generate code from models. The ultimate goal of research, which this thesis if part of, is to create AADL/BLESS to SPARK/Ada traslator. Ultimatelly there will be standardized AADL/BLESS models, which will be generating code base for developers extensions (like sceleton code for some Web Framework).

This thesis propose mapping from AADL/BLESS to SPARK/Ada. As an example of Medical Device, PCA Pump (Patient Controlled Analgesia) is used. The foundation for this work is System Requirements for "Integrated Clinical Environment Patient-Controlled Analgesia Infusion Pump System Requirements" (DRAFT 0.10.1) \cite{PcaReq} and AADL Models with BLESS annexes created by Brian Larson. Additionally, there was a contribution made in clarifying the requirements document and extending AADL models.
