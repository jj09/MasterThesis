%!TEX root = JakubJedryszek-MasterThesis.tex
% +--------------------------------------------------------------------+
% | Abstract Page
% +--------------------------------------------------------------------+

\pagestyle{empty}
%\vspace{1cm}
\setlength{\baselineskip}{0.8cm}

\indent

% +--------------------------------------------------------------------+
% | Enter the text of your abstract below, maximum of 350 words.
% +--------------------------------------------------------------------+

Medical devices are safety-critical systems whose failure may put human life in danger. They are becoming more advanced and thus more complex. This leads to bigger and more complicated code-bases that are hard to maintain and verify. Model-driven development provides high-level and abstract description of the system in the form of models that omit details, which are not relevant during the design phase. This allows for certain types of verification and hazard analysis to be performed on the models. These models can then be translated into code. However, errors that do not exist in the models may be introduced during the implementation phase. Automated translation from verified models to code may prevent to some extent.

This thesis proposes approach for model-driven development and verification of medical devices. Models are created in AADL (Architecture Analysis \& Design Language), a language for software and hardware architecture modeling. AADL models are translated to SPARK Ada, contract-based programming language, which is suitable for software verification. Generated code base is further extended by developers to implement internals of specific devices. Created programs can be verified using SPARK tools.

A PCA (Patient Controlled Analgesia) pump medical device is used to illustrate the primary artifacts and process steps. The foundation for this work is "Integrated Clinical Environment Patient-Controlled Analgesia Infusion Pump System Requirements" document and AADL Models created by Brian Larson. In addition to proposed model driven development approach, a PCA pump prototype was created using the BeagleBoard-xM device as a platform. Some components of PCA pump prototype were verified by SPARK tools and Bakar Kiasan.
