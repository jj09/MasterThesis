%!TEX root = JakubJedryszek-MasterThesis.tex
% +--------------------------------------------------------------------+
% | Abstract Page
% +--------------------------------------------------------------------+

\pagestyle{empty}
%\vspace{1cm}
\setlength{\baselineskip}{0.8cm}

\indent

% +--------------------------------------------------------------------+
% | Enter the text of your abstract below, maximum of 350 words.
% +--------------------------------------------------------------------+

Nowadays, medical devices works rather independently. It leads to many accidents, which could have been avoided by their interoperability. For example some drug (e.g. morphine), which is delivered by Patient-controlled analgesia (PCA) pump after surgery, can cause low oxygen level or even lack of pulse. That can lead to patient's death. PCA pump does not monitor oxygen level, but Oxygen monitoring device does. If these two devices are organized in centralized system, which implements safety interlock mechanism to shutdown the pump, accident can be avoided. 

In order to communicate, devices have to use compatible interfaces and protocols. There is a concept of "Integrated Clinical Environment" (ICE). It is series of standards, which describes medical device integration and interoperability. SAnToS lab created Medical Device Coordination Framework (MDCF), which is prototype implementation of ICE. Standards are captured not only as requirement documents, but also in software and hardware models form. It allows different medical devices, created by different vendors, to be connected and work under supervision of centralized system.

This thesis propose approach for model-driven development and verification of medical devices. Models are created in AADL (Architecture Analysis \& Design Language), language for software and hardware architecture modeling. AADL models are translated to SPARK Ada, contract-based programming language, which is suitable for software verification. Generated code base is further extended by developers to implement internals of specific devices. Created programs can be verified using SPARK tools.

As an example of medical device, PCA (Patient Controlled Analgesia) pump is used. The foundation for this work is "Integrated Clinical Environment Patient-Controlled Analgesia Infusion Pump System Requirements" document \cite{PcaReq} and AADL Models created by Brian Larson. In addition to proposed mapping, PCA pump prototype was created. As a platform for prototyping, BeagleBoard-xM device was used. Some components of PCA pump prototype are verified by SPARK tools and Bakar Kiasan.
