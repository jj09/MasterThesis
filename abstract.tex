%!TEX root = JakubJedryszek-MasterThesis.tex
% +--------------------------------------------------------------------+
% | Abstract Page
% +--------------------------------------------------------------------+

\pagestyle{empty}
%\vspace{1cm}
\setlength{\baselineskip}{0.8cm}

\indent

% +--------------------------------------------------------------------+
% | Enter the text of your abstract below, maximum of 350 words.
% +--------------------------------------------------------------------+

Nowadays, medical devices works rather independently. It leads to many accidents, which could have been avoided by their interoperability. E.g. drug, which is delivered by infusion pump can cause low oxygen level or even lack of pulse and ultimately patient's death. Infusion pump cannot detect it. However, Oxygen monitoring device can. If these two devices were connected and centralized system could make a decision of infusion pump shutdown, accident could have been avoided. 

In order to communicate, devices have to use standardized interfaces and protocols. Dr. Julian Goldman developed idea of "Integrated Clinical Environment" (ICE). It is series of standards, which describes medical device interoperability. SAnToS lab created Medical Device Coordination Framework (MDCF), which is prototype implementation of ICE. Standards are captured not only as requirement documents, but also as software and hardware models. Based on these standards, different medical devices can be created and connected. Devices are monitored and controlled by MDCF controller.

Nowadays, there is a trend to generate code from models. The ultimate goal of research, which this thesis is part of, is to create translator, which automatically translates medical device models into code. Generated code base would be further extended by developers to implement internals of specific devices. 

Models are created in AADL (Architecture Analysis \& Design Language). Target programming language is SPARK Ada. Ultimately from set of standardized AADL models, SPARK Ada code would be generated. 

This thesis propose mapping from AADL models to SPARK Ada. As an example of Medical Device, PCA (Patient Controlled Analgesia) pump is used. The foundation for this work is "Integrated Clinical Environment Patient-Controlled Analgesia Infusion Pump System Requirements" document \cite{PcaReq} and AADL Models created by Brian Larson. In addition to proposed mapping, PCA pump prototype was created. As a platform for prototyping, BeagleBoard-xM device was used. Some components of PCA pump prototype are verified by SPARK tools and Bakar Kiasan.
