% +--------------------------------------------------------------------+
% | LaTeX Template                                                     |
% | for K-State Electronic Theses, Dissertations, and Reports          |
% |                                                                    |
% | Comments and guidelines for using the template are shown           |
% | within boxes like this one.                                        |
% |                                                                    |
% | Revised 6/30/06                                                    |
% | 9/14/06: Removed typos                                             |
% | 3/29/13: Commented out hypernat package                            |
% | 4/5/13: Changed to plain bib style
% | 5/17/13: added /cleardoublepage and /phantomsection to
% |          /bibliography to correct TOC page problem
% | 5/17/13: Fixed TOC problem with Dedication, Preface, etc.          |
% +--------------------------------------------------------------------+

% +--------------------------------------------------------------------+
% | Your paper should contain the following sections, except where     |
% | indicated as optional, in the order shown.  Also, all headings     |
% | shown with an asterisk (*) must be centered and in uppercase       |
% | letters:                                                           |
% |                                                                    |
% | Abstract Title Page (doctoral dissertations only)                  |
% | ABSTRACT* (doctoral dissertations only)                            |
% | Title Page                                                         |
% | Copyright Page (Optional - only needed if copyrighting)            |
% | ABSTRACT *                                                         |
% | TABLE OF CONTENTS *                                                |
% | LIST OF FIGURES *                                                  |
% | LIST OF TABLES*                                                    |
% | ACKNOWLEDGMENTS* (Optional)                                        |
% | DEDICATION * (Optional)                                            |
% | PREFACE * (Optional)                                               |
% | Individual Chapters                                                |
% | References and/or bibliography                                     |
% | Appendices (as needed)                                             |
% +--------------------------------------------------------------------+

% +--------------------------------------------------------------------+
% | The LaTex keyword \documentclass selects a particular class to     |
% | associate with the document.  The current documentclass            |
% | {class_diss} generates a Table of Contents that has leading dots   |
% | only on chapter subheadings.  If you prefer a Table of Contents    |
% | that has leading dots for all entries, replace {class_diss}        |
% | with {Mydiss} in the command below.                                |
% |                                                                    |
% +--------------------------------------------------------------------+

\documentclass[final, 12pt,oneside]{class_diss}

% +--------------------------------------------------------------------+
% | The bibliography style is set to a generic superscript. Other      |
% | styles are available in the styles directory.  To use an           |
% | author/year style, you'll need to make several adjustments:        |
% |   1.  In the \bibliographystyle command below, replace "unsrtnat"  |
% |       with the desired style from the \styles directory, e.g.,     |
% |       \bibliographystyle{styles\apa}                               |
% |   2.  In the \bibpunct command (several lines below), change the   |
% |       "s" to an "a"                                                |
% |   3.  Use "\citep" rather than "\cite" when making a citation in   |
% |       the text.                                                    |
% +--------------------------------------------------------------------+

\bibliographystyle{alpha}

% +--------------------------------------------------------------------+
% | Now, we add in all external packages that we will use throughout   |
% | the document.  You can add other packages as needed.
% +--------------------------------------------------------------------+

%\usepackage{     caption2} % Customize captions a bit more
\usepackage{      amsmath} % American Mathematics Society standards
%\usepackage{      wrapfig} % Wraps text around a figure or table
\usepackage{     graphicx} % Extended graphics package.
%\usepackage{     fancyhdr} % Efficiently handles headers and footers
%\usepackage{       braket} % Bra-Ket notation package
%\usepackage{     mathrsfs} % Specialized Math fonts (Hamiltonian, etc.)
%\usepackage{boxedminipage} % Boxed text can be produced
%\usepackage{     setspace} % Controls line spacing via \begin{space}

\usepackage{amsxtra}
\usepackage{amssymb}
\usepackage{amsthm}
\usepackage{latexsym}
\usepackage{setspace}
\usepackage[T1]{fontenc}    % Added by Jakub Jedryszek to support < and > in text
\usepackage[utf8]{inputenc} % Added by Jakub Jedryszek to support non-English letters
\usepackage{longtable}      % Added by Jakub Jedryszek to support tables bigger than 1 page


% +--------------------------------------------------------------------+
% | The color package allows one to select colors for hyperlinking     |
% | (see below).                                                       |
% +--------------------------------------------------------------------+

\usepackage[usenames]{color}

% +--------------------------------------------------------------------+
% | Colors defined for use with this template.                         |
% +--------------------------------------------------------------------+

\definecolor{Pink}{rgb}{1.0, 0.5, 0.5}
\definecolor{Maroon}{rgb}{0.8, 0.0, 0.0}

% +--------------------------------------------------------------------+
% | Added by Jakub Jedryszek for code snippets                         |
% | http://en.wikibooks.org/wiki/LaTeX/Source_Code_Listings            |
% +--------------------------------------------------------------------+

\usepackage{listings}

\definecolor{mygreen}{rgb}{0,0.6,0}
\definecolor{mygray}{rgb}{0.5,0.5,0.5}
\definecolor{red}{rgb}{0.6,0,0}
\definecolor{mymauve}{rgb}{0.58,0,0.82}

% Support syntax highlighting for AADL

\lstdefinelanguage{aadl}
{
    morekeywords = {in,out,package,end,bus,data,thread,port,group,process,processor,
        system,memory,device,subprogram,public,private,event,property,set,applies,to,
        units,type,implementation,parameter,reference},
    morekeywords = {properties,features,annex,modes,connections,flows,
        subcomponents,calls,binding},
    morekeywords = {aadlinteger,aadlboolean,aadlstring,aadlfloat},
    morecomment = [l]{--},
}

\lstdefinelanguage{bless}
{
    morekeywords = {all,skip,fetchadd,computation,availability,spread,swap,real,do,numberof,D,stop,invariant,not,integer,now,string,mod,d,iff,timeout,exists,implies,are,transitions,xor,for,fetchxor,sum,state,rational,rem,shared,C,bold,variant,post,throw,c,cor,of,nonvolatile,forall,or,pre,variables,bound,boolean,false,array,initial,.,type,final,on dispatch,B,complete,that,assert,catch,e,true,count,b,f,until,record,while,declare,def,and,cand,constant,states,in,null,setmode,if,fetchor,in mode,F,when,enumeration,complex,units,A,product,E,tops,fi,a,natural,modes,fetchand,fresh},
    morecomment = [l]{--},
}

\lstdefinelanguage{ada2012}
{
    morekeywords = {abort, else, new, return, abs, elsif, not, reverse, abstract, end, null, accept, entry, select, access, exception, of, separate, aliased, exit, or, some, all, others, subtype, and, for, out, synchronized, array, function,     overriding, at, tagged, generic, package, task, begin, goto, pragma, terminate, body, private, then, if, procedure, type, case, in, protected, constant, interface, until, is, raise, use, declare, range, delay, limited, record, when, delta, loop, rem, while, digits, renames, with, do, mod, requeue, xor},
    morecomment = [l]{--},   
}

% Layout for listings

%\lstset{language=aadl,
%        basicstyle=\scriptsize\sffamily,
%        aboveskip=.1cm, % \smallskipamount, % \bigskipamount,
%        belowskip= \smallskipamount, % \bigskipamount,
%        abovecaptionskip=-.5cm, % \smallskipamount, % \medskipamount,
%        belowcaptionskip=.0cm, % \smallskipamount, % \bigskipamount,
%        xleftmargin=.0cm,
%        captionpos=b,
%        tabsize=3,
%        }

\lstset{ %
  backgroundcolor=\color{white},   % choose the background color; you must add \usepackage{color} or \usepackage{xcolor}
  basicstyle=\footnotesize\ttfamily,        % the size of the fonts that are used for the code
  breakatwhitespace=false,         % sets if automatic breaks should only happen at whitespace
  breaklines=true,                 % sets automatic line breaking
  captionpos=b,                    % sets the caption-position to bottom
  columns=fullflexible,
  commentstyle=\color{mygreen},    % comment style
  %deletekeywords={...},            % if you want to delete keywords from the given language
  escapeinside={\%*}{*)},          % if you want to add LaTeX within your code
  extendedchars=true,              % lets you use non-ASCII characters; for 8-bits encodings only, does not work with UTF-8
  %frame=single,                    % adds a frame around the code
  keepspaces=true,                 % keeps spaces in text, useful for keeping indentation of code (possibly needs columns=flexible)
  keywordstyle=\bf,       % keyword style
  language=Ada,                    % the language of the code
  %morekeywords={*,...},            % if you want to add more keywords to the set
  numbers=none,                    % where to put the line-numbers; possible values are (none, left, right)
  numbersep=2,                   % how far the line-numbers are from the code
  numberstyle=\tiny\color{mygray}, % the style that is used for the line-numbers
  rulecolor=\color{black},         % if not set, the frame-color may be changed on line-breaks within not-black text (e.g. comments (green here))
  showspaces=false,                % show spaces everywhere adding particular underscores; it overrides 'showstringspaces'
  showstringspaces=false,          % underline spaces within strings only
  showtabs=false,                  % show tabs within strings adding particular underscores
  stepnumber=2,                    % the step between two line-numbers. If it's 1, each line will be numbered
  stringstyle=\color{red},     % string literal style
  tabsize=2,                       % sets default tabsize to 2 spaces
  %title=\lstname                   % show the filename of files included with \lstinputlisting; also try caption instead of title
  gobble=6
}

% +--------------------------------------------------------------------+
% | In the commands below, we use the 'natbib' package, and specify    |
% | the 'sort&compress' option, which condenses                        |
% | citations from (1,2,3,5,9,10,11) to (1-3,5,9-11).  The 'bibpunct'  |
% | option selects various parameters for how the citation will be     |
% | displayed.  In this case, only the comma (separation between       |
% | citations) and the 's' (superscript) arguments are chosen.  The    |
% | other curly braces deal with how to 'wrap' the citation (using     |
% | parentheses, brackets, etc.) and are not needed for the chosen     |
% | style.                                                             |
% +--------------------------------------------------------------------+

\usepackage{natbib}
\bibpunct{[}{]}{,}{n}{}{}
%\usepackage{hypernat}

% +--------------------------------------------------------------------+
% | Lastly, the hyperref package allows one to hyperlink cross-        |
% | references and figures in a LaTeX document.                        |
% | 3/29/13 - Hypernat package commented out because  it is no longer      |
% | needed with later versions of hyperref and natbib.                 |
% +--------------------------------------------------------------------+

\usepackage[pdftex, plainpages=false, pdfpagelabels]{hyperref}

\hypersetup{
    linktocpage=true,
    colorlinks=true,
    %bookmarks=true,
    citecolor=blue,
    urlcolor=red,
    linkcolor=Maroon,
    citebordercolor={1 0 0},
    urlbordercolor={1 0 0},
    linkbordercolor={.7 .8 .8},
    breaklinks=true,
    %pdfpagelabels=true,
    }

% +--------------------------------------------------------------------+
% | Page margins are set on 1 inch on all sides.                       |
% +--------------------------------------------------------------------+

\topmargin      = -0.56in
\textheight     =  8.60in
\textwidth      =  6.46in
\oddsidemargin  =  0.02in


% +--------------------------------------------------------------------+
% | The document finally begins here.                                  |
% +--------------------------------------------------------------------+

\doublespacing
\begin{document}

% +--------------------------------------------------------------------+
% | Masters Students -- You Need to Make Some Changes Here

% | The Abstract Title page and Abstract following the Abstract Title
% | page are required only for doctoral dissertations.  For masters
% | theses or reports, comment out or delete the following 7 lines:
% | \input{abstracttitle.tex} through \end{abstract}.  You will also
% | need to uncomment two lines under "Abstract" below.
% |
% +--------------------------------------------------------------------+

%\input{abstracttitle.tex}
%
%\begin{abstract}
%   \setcounter{page}{-1}
%   \pdfbookmark[0]{Abstract}{PDFAbstractPage}
%   %!TEX root = etdrtemplate.tex
% +--------------------------------------------------------------------+
% | Abstract Page
% +--------------------------------------------------------------------+

\pagestyle{empty}
%\vspace{1cm}
\setlength{\baselineskip}{0.8cm}

\indent

% +--------------------------------------------------------------------+
% | Enter the text of your abstract below, maximum of 350 words.
% +--------------------------------------------------------------------+

The future of Medical Devices is their interoperability. Nowadays, medical devices works rather independently, which cause accidents we could avoid if different devices would be able to communicate. Dr. Julian Goldman developed idea of "Integrated Clinical Environment" (ICE). It is series of standards, which describes medical device interoperability. SAnToS lab created Medical Device Coordination Framework (MDCF), which is implementation of ICE idea.

Ada is one of the most popular (along with C/C++) programming language targeted at embedded and real-time systems. SPARK Ada is subset of Ada, designed for the development of safety and security critical systems. It contains properties, which allows to prove correctness of program and its entities.

AADL (Architecture Analysis \& Design Language) is modeling language for representing hardware and software. It is used for real-time, safety critical and embedded systems. AADL allows for the description of both software and hardware parts of a system. It is used to describe architecture, but AADL it allows behavioral extensions through annex languages. BLESS (Behavior Language for Embedded Systems with Software) is AADL annex sub language defining behavior of components. The goal of BLESS is automatically-checked correctness proofs of AADL models of embedded electronic systems with software.

Nowadays, there is a trend to generate code from models. The ultimate goal of research, which this thesis if part of, is to create AADL/BLESS to SPARK Ada translator. Ultimately there will be standardized AADL/BLESS models, from which SPARK Ada code base will be generated. It will be starting point for developers, who will implement and extend it.

This thesis propose mapping from AADL/BLESS to SPARK Ada. As an example of Medical Device, PCA (Patient Controlled Analgesia) Pump is used. The foundation for this work is System Requirements for "Integrated Clinical Environment Patient-Controlled Analgesia Infusion Pump System Requirements" (DRAFT 0.10.1) \cite{PcaReq} and AADL Models with BLESS annexes created by Brian Larson. Additionally PCA Pump prototype was created. As a platform for prototyping, BeagleBoard-xM device was used.

%   \vfill
%\end{abstract}

% +--------------------------------------------------------------------+
% | Title Page -- Required for both Doctoral and Masters Students
% +--------------------------------------------------------------------+

\input{title.tex}

% +--------------------------------------------------------------------+
% | Copyright Page -- Required for both Doctoral and Masters Students
% +--------------------------------------------------------------------+

\input{copyright.tex}

% +--------------------------------------------------------------------+
% |  Abstract -- Required for both Doctoral and Masters Students
% +--------------------------------------------------------------------+

\begin{abstract}

% +--------------------------------------------------------------------+
% | For masters theses or reports, uncomment the following commands:
% +--------------------------------------------------------------------+

   \setcounter{page}{-1}
   \pdfbookmark[0]{Abstract}{PDFAbstractPage}

    %!TEX root = etdrtemplate.tex
% +--------------------------------------------------------------------+
% | Abstract Page
% +--------------------------------------------------------------------+

\pagestyle{empty}
%\vspace{1cm}
\setlength{\baselineskip}{0.8cm}

\indent

% +--------------------------------------------------------------------+
% | Enter the text of your abstract below, maximum of 350 words.
% +--------------------------------------------------------------------+

The future of Medical Devices is their interoperability. Nowadays, medical devices works rather independently, which cause accidents we could avoid if different devices would be able to communicate. Dr. Julian Goldman developed idea of "Integrated Clinical Environment" (ICE). It is series of standards, which describes medical device interoperability. SAnToS lab created Medical Device Coordination Framework (MDCF), which is implementation of ICE idea.

Ada is one of the most popular (along with C/C++) programming language targeted at embedded and real-time systems. SPARK Ada is subset of Ada, designed for the development of safety and security critical systems. It contains properties, which allows to prove correctness of program and its entities.

AADL (Architecture Analysis \& Design Language) is modeling language for representing hardware and software. It is used for real-time, safety critical and embedded systems. AADL allows for the description of both software and hardware parts of a system. It is used to describe architecture, but AADL it allows behavioral extensions through annex languages. BLESS (Behavior Language for Embedded Systems with Software) is AADL annex sub language defining behavior of components. The goal of BLESS is automatically-checked correctness proofs of AADL models of embedded electronic systems with software.

Nowadays, there is a trend to generate code from models. The ultimate goal of research, which this thesis if part of, is to create AADL/BLESS to SPARK Ada translator. Ultimately there will be standardized AADL/BLESS models, from which SPARK Ada code base will be generated. It will be starting point for developers, who will implement and extend it.

This thesis propose mapping from AADL/BLESS to SPARK Ada. As an example of Medical Device, PCA (Patient Controlled Analgesia) Pump is used. The foundation for this work is System Requirements for "Integrated Clinical Environment Patient-Controlled Analgesia Infusion Pump System Requirements" (DRAFT 0.10.1) \cite{PcaReq} and AADL Models with BLESS annexes created by Brian Larson. Additionally PCA Pump prototype was created. As a platform for prototyping, BeagleBoard-xM device was used.

    \vfill

\end{abstract}

% +--------------------------------------------------------------------+
% | We use the following code to suppress page numbers and other
% | style issues we do not want present on a given page.               |
% +--------------------------------------------------------------------+

%\thispagestyle{empty} Looks like it's ok to remove this line
\newpage
\pagenumbering{roman}

% +--------------------------------------------------------------------+
% | On the line below, set the number to represent the page number of
% | the Table of Contents page.  For example, if the Table of Contents
% | page is the 8th page of your document, enter 8 in the brackets.  This
% | number may vary, depending on the length of your abstract.
% |
% | Numbers do not appear on the title and abstract pages, but they are
% | included in the page count.  The Table of Contents page is the
% | first page on which page numbers are displayed.
% +--------------------------------------------------------------------+

\setcounter{page}{8}

% +--------------------------------------------------------------------+
% | Here, we will generate our Table of Contents (TOC) entries.        |
% | This adds the section to the TOC and then generates the indicated  |
% | section.                                                           |
% +--------------------------------------------------------------------+

\phantomsection
\addcontentsline{toc}{chapter}{Table of Contents}

\tableofcontents
\listoffigures
\listoftables

%\hfill  Are these lines necessary?
%\hfill

% +--------------------------------------------------------------------+
% | Acknowledgements Page
% |
% | If you choose not to have an Acknowledgements page, comment out
% | or delete the following 3 lines.
% +--------------------------------------------------------------------+

\phantomsection
\addcontentsline{toc}{chapter}{Acknowledgements}
% +--------------------------------------------------------------------+
% | Acknowledgements Page (Optional)                                   |
% +--------------------------------------------------------------------+

\newpage
\vspace*{0.9cm}
\begin{center}
{\bf \Huge Acknowledgments}
\end{center}

\setlength{\baselineskip}{0.8cm}

%\pdfbookmark[0]{Acknowledgements}{PDF_Acknowledgements}

% +--------------------------------------------------------------------+
% | Enter text for your acknowledgements in the space below this box.  |
% |                                                                    |
% +--------------------------------------------------------------------+

Say thank you for everybody involved directly and indirectly.


% +--------------------------------------------------------------------+
% | Dedication Page
% |
% | If you choose not to have a Dedication page, comment out
% | or delete the following 3 lines.
% +--------------------------------------------------------------------+

\phantomsection
\addcontentsline{toc}{chapter}{Dedication}
%!TEX root = JakubJedryszek2014.tex
% +--------------------------------------------------------------------+
% | Dedication Page (Optional)
% +--------------------------------------------------------------------+

\newpage
\vspace*{0.9cm}
\begin{center}
{\bf \Huge Dedication}
\end{center}


\setlength{\baselineskip}{0.8cm}

%\pdfbookmark[0]{Dedication}{PDF_Dedication}
\phantomsection
\addcontentsline{toc}{chapter}{Dedication}

\newenvironment{dedication}
  {%\clearpage           % we want a new page
   %\thispagestyle{empty}% no header and footer
   \vspace*{\stretch{1}}% some space at the top
   \itshape             % the text is in italics
   \raggedleft          % flush to the right margin`'
  }
  {\par % end the paragraph
   \vspace{\stretch{3}} % space at bottom is three times that at the top
   \clearpage           % finish off the page
  }

\begin{dedication}
	\begin{adjustwidth}{8em}{8em}
		\begin{center}
			For my family, mentors, friends and all people \\
			who inspired me directly or indirectly \\
			in things I do.
		\end{center}
	\end{adjustwidth}
\end{dedication}




% +--------------------------------------------------------------------+
% | Preface Page
% +--------------------------------------------------------------------+

%\addcontentsline{toc}{chapter}{Preface}
%\input{preface.tex}

\phantomsection
% +--------------------------------------------------------------------+
% | We use arabic (1, 2, 3...) page numbering starting from page 1.    |
% | Note, however, that there are many pages where this is not the     |
% | desired behavior - such as the Title page, or abstract.  In these  |
% | cases, we can use \thispagestyle{empty} to suppress page numbers,  |
% | and other general style issues that we've defined globally.        |
% +--------------------------------------------------------------------+

\newpage
\pagenumbering{arabic}
\setcounter{page}{1}

% +--------------------------------------------------------------------+
% | Here is where we include individual sections of the thesis or
% | dissertation.                                                      |
% +--------------------------------------------------------------------+

% +--------------------------------------------------------------------+
% | Chapters
% +--------------------------------------------------------------------+

%!TEX root = etdrtemplate.tex
% +--------------------------------------------------------------------+
% | Chapter 1
% +--------------------------------------------------------------------+

\cleardoublepage

% +--------------------------------------------------------------------+
% | Replace "Chapter Title" below with the title of your chapter.  LaTeX
% | will automatically number the chapters.
% +--------------------------------------------------------------------+

\chapter{Introduction}
\label{introduction}

The tale about software safety: why important, software everywhere, human life, etc. (info from 890).

Software Engineering for Real-Time and Safety-Critical systems is very different than creating Desktop applications. In both types of software we want to ensure correctnes and security. In case of e.g. e-mail client software assurance is not crucial. When something happend, we just restart the app. However, in case of e.g. Airplane, software cannot just crash. If it crashes, then people dies. Behind these reasons, we need different properties of our programming language and its tools. For Web or Mobile apps our priority is Rapid Development. For Safety-Critical systems, security is crucial.


Most important in Safety-Critical Systems: Hazard analysis (avoid, recover)!
Hazard can cause:
	* Incident
	* Accident
Accident - event, which cause loss (undesired)
Incident - event, which not cause loss (but undesired)
Hazard + Environmental Conditions => Accident (loss)
Event - state change


\section{Motivation}
\label{introduction:motivation}
There are many accidents where Medical Devices are involved. Very often, the reason is the lack of communication between different Medical Devices. [EXAMPLE ACCIDENT]
The solution for such a problem is to create "Integrated Clinical Environment" (ICE). SAnToS Lab at Kansas State University is working on Medical Device Coordination Framework (MDCF), which is prototype implementation of ICE.  

Devices working under MDCF will need to satisfy some requirements. To make Developer's life easier, the requirements will be not only in documentation, but also in code. The code will be generated from models.
Model Driven Development in this case means we will have some base models for medical devices development and developer will extend and customize them. The same like you do File > 'New Java project' in Eclipse, we want to be able to do the same in e.g. GNAT Programming Studio: File > 'New Medical device project'.
Model as specification/requirements.

PCA Pump is as an example of Medical Device, which ultimately will work under Medical Device Coordination Framework (MDCF) developed by SAnToS Lab at Kansas State University.
Summarizing, we want to be able to have MDCF, which coordinates Medical Devices. Additionally we want set of AADL/BLESS models, which can be automatically translated to SPARK Ada. These models will be base for Medical Devices Developers, who can extend and adjust them to implement specific devices. 
Why AADL? Because it describes hardware and software. It allows to validate that the software will work on some device.
Why SPARK? Because it contains set of verification tools. 
Testing vs Verification (form 721 slides): Testing starts with a set of possible test cases, simulates the system on each input, and observes the behavior. In general, testing does not cover all possible executions. On the other hand, verification establishes correctness for all possible execution sequences.
Techniques for Verification:
\begin{itemize}
	\item Formal verification: prove mathematically that the program is correct – this can be difficult for large programs.
	\item Correctness by construction: follow a well- defined methodology for constructing programs.
	\item Model checking: enumerate all possible executions and states, and check each state for correctness.
\end{itemize}
SPARK is a subset of Ada language, which is easy to deal with it. In the future, when everything will be done (in case of proving perspective) in SPARK, it will (probably) be extended. Maybe finally, there will be no SPARK, but only Ada. Thus for now, SPARK is temporary subset of Ada for reasoning and corectness proving.


\section{Goals}
\label{introduction:goals}
\begin{itemize}
	\item learn about PCA Pump Infusion pumps properties
	\item SPARK Ada crosscompilation for ARM-device (BeagleBoard-xM)
	\item implement PCA Pump based on Brian's Requirement Document (using Ravenscar profile)
	\item propose AADL/BLESS to SPARK Ada mapping
	\item mock PCA Pump AADL/BLESS models in SPARK Ada
	\item implement not generated part
	\item create AADL/BLESS to SPARK Ada translator
	\item Use SPARK toolset for software verification:
		\begin{itemize}
			\item SPARK Examiner
			\item GNATprove
			\item Sireum Kiasan
			\item Sireum Alir?
		\end{itemize}
\end{itemize}




\section{Contribution}
\label{introduction:contribution}
Put all piecies together (SPARK, AADL, BLESS?, ICE, PCA Pump) and analyze current state of target technologies.
Review PCA Pump Requirements document
Implement PCA Pump based on document, by resolving ambiguities and analyzing different implementation possibilities. Then the implementation is sort of proof that, this document might be base for future Infusion Pumps and/or Medical Devices implementations.
Analyzed and extended PCA Pump AADL models, then based on available resource propose possible translation from AADL/BLESS to SPARK Ada.
Created AADL/BLESS to SPARK Ada translator?
Showed use of SPARK toolset.


\section{Organization}
\label{introduction:organization}
The thesis is organized in \ref{future_work} chapters. 
Chapter \ref{introduction} is the of the problem and summary of contribution which was made. 
Chapter \ref{background} is Background that gives details about Model Driven Development, SPARK Ada, AADL/BLESS, ICE and available tools for such environment. 
Chapter \ref{pcapump} describes the implementation of PCA Pump Prototype. Faced issues and design decisions made.
Chapter \ref{codegen} is about code generation from the model. 
Chapter \ref{summary} summarizes all work which has been done in this thesis. 
Chapter \ref{future_work} is the future work that can be done on this topic.


\section{Terms and Acronyms}
\label{introduction:terms}

\begin{itemize}
	\item \textbf{AADL} - Architecture Analysis \& Design Language
	\item \textbf{BLESS} - Behavioral Language for Embedded Systems with Software
	\item \textbf{ICE} - Integrated Clinical Environment
	\item \textbf{MDCF} - Medical Device Coordination Framework
	\item \textbf{PCA} - Patient-Controlled Analgesia (pump)
	\item \textbf{AADL} - Architecture Analysis \& Design Language
	\item \textbf{AADL} - Architecture Analysis \& Design Language
	\item \textbf{AADL} - Architecture Analysis \& Design Language
	\item \textbf{AADL} - Architecture Analysis \& Design Language
	\item \textbf{AADL} - Architecture Analysis \& Design Language
\end{itemize}

% +--------------------------------------------------------------------+
% | Uncomment the lines below to add additional chapters.  Name the
% | files chapter2.tex for Chapter 2, chapter3.tex for Chapter 3, etc.
% +--------------------------------------------------------------------+

% +--------------------------------------------------------------------+
% | Sample Chapter 2
% +--------------------------------------------------------------------+

\cleardoublepage

% +--------------------------------------------------------------------+
% | Replace "This is Chapter 2" below with the title of your chapter.
% | LaTeX will automatically number the chapters.
% +--------------------------------------------------------------------+

\chapter{Background}
\label{background}

General overview of all below things?

\section{High-assurance systems}
\label{background:highas}
What software properties and requirements exists to ensure safety.
Medical Devices as type of High-assurance systems.


\section{SPARK/Ada}
\label{background:spark}
History of Ada: http://www.adahome.com/History/Steelman/intro.htm 
Ada fulfill US DOD requirements.
Rationale of this language. Good for Software Verification etc.
http://www.slideshare.net/AdaCore/ada-2012 (slide 11: dev responsibility)
GNAT Programming Studio.
Tools for corectness proving.

\subsection{Sireum Kiasan}
\label{background:spark:sireum}
Overview: symbolic execution, Pilar, Alir?
Sireum Kiasan[~\cite{Sireum:RobbyICSE13,Sireum:Kiasan1,Sireum:Kiasan2}] is a tool, which use symbolic execution for finding possible paths in program.
Plugin for GNAT Programming Studio.
Plugin for Eclipse (but only SPARK 2005).

\subsection{GNAT Prove}
\label{background:spark:gnatprove}
Overview

\subsection{AUnit}
\label{background:spark:aunit}
Overview


\section{AADL}
\label{background:aadl}
Rationale of this language.
%https://wiki.sei.cmu.edu/aadl/index.php/The_Story_of_AADL


\section{BLESS}
\label{background:bless}
How it fits into the picture. Why it was developed. Corectness prove in AADL + bahavior, from which we can generate SPARK/Ada code.


\section{Integrated Clinical Environment}
\label{background:ice}
http://santos.cis.ksu.edu/MDCF/doc/ICE-Motivation.pdf
http://santos.cis.ksu.edu/MDCF/doc/MDCF-Tutorial-Overview.pdf


\section{PCA Pump}
\label{background:pcapump}
http://www.santoslab.org/pub/paper/LarsonEtAl13-PCA-Requirements-SEHC-preprint.pdf


\section{AADL/BLESS to SPARK/Ada code generation}
\label{background:codegen}
The ultimate goal of long term research, this thesis is part of.
AADL->Ada
BLESS->SPARK contracts + (eventually) behavior

\subsection{Ocarina}
\label{background:codegen:ocarina}
Overview: AADL->Ada on PolyOrb, support for other languages?

\subsection{Ramses}
\label{background:codegen:ramses}
Overview. Not sure if it is useful at any point here?
%!TEX root = JakubJedryszek-MasterThesis.tex

\cleardoublepage


\chapter{PCA Pump}
\label{pcapump}

% http://www.santoslab.org/pub/paper/LarsonEtAl13-PCA-Requirements-SEHC-preprint.pdf
% http://ppahs.org/2012/05/30/patient-controlled-analgesia-pca-pumps-the-basics/

\begin{wrapfigure}{r}{0.4\textwidth}
  \begin{center}
    \includegraphics[width=0.4\textwidth]{figures/pca-pump.png}
  \end{center}
  \caption{Patient Controlled Analgesia (PCA) pump}
  \label{figure:pca-pump}
\end{wrapfigure}

Patient Controlled Analgesia (PCA) pump is a medical device, which allows the patient to self-administer small doses of narcotics (usually Morphine, Dilaudid, Demerol, or Fentanyl). PCA pumps are commonly used after surgery to provide a more effective method of pain control than periodic injections of narcotics. A continuous infusion (called a basal rate) permits the patient to receive a continuous infusion of pain medication. There is no need for a clinician to administer it. Patient can also request additional boluses, but only in specified intervals. It prevents from over infusion. In addition to basal and patient bolus, clinician can also request bolus called clinician bolus or square bolus. 

Figure \ref{figure:pca-pump} shows LifeCare PCA pump. On the left hand side, there is drug reservoir. On the right -  clinician panel, which allows to control the pump. Figure \ref{figure:alaris-pump} shows PCA Pump, made by company Alaris. 

\begin{wrapfigure}{l}{0.4\textwidth}
  \begin{center}
    \includegraphics[width=0.4\textwidth]{figures/alaris-pump.png}
  \end{center}
  \caption{Alaris Pump}
  \label{figure:alaris-pump}
\end{wrapfigure}

PCA Pump is safety-critical device which works in standard process control loop depicted in the figure \ref{figure:control-loop}. The controller obtains information about (observes) the process state from measured variables (feedback) and uses this information to initiate action by manipulating controlled variables to keep the process operating within predefined limits or set points (the goal) despite disturbances to the process. Such as different air pressure or device position (gravity impact). In general, the maintenance of any open-system hierarchy (either biological or man-made) will require a set of processes in which there is communication of information for regulation or control. \cite{SaferWorld} 

\begin{figure}[ht]%t=top, b=bottom, h=here
    \begin{center}
    	\includegraphics[width=0.9\textwidth]{figures/safety-critical-loop.png}    	
    \end{center}
    \caption{Standard Process Control Loop.}
    \label{figure:control-loop}
\end{figure}

PCA Pump actuator is motor, which pump drug to the patient's vein. Controlled process is dosing the drug. Sensors measure amount of dosed drug. They might be used for double-check if ordered (by controller) amount of drug was appropriately delivered. Sometimes there might be some distrubances caused by mechanical issues and environmental conditions. Controller issues appropriate actions based on informations from sensors and clinician or patient's commands. High level overview of PCA Pump is depicted in the figure \ref{figure:pca-pump-system}.

\begin{figure}[ht]%t=top, b=bottom, h=here
    \begin{center}
    	\includegraphics[width=0.9\textwidth]{figures/pca-pump-system.png}    	
    \end{center}
    \caption{PCA Pump system}
    \label{figure:pca-pump-system}
\end{figure}

One of the hazards of using PCA pumps, is that there is inadequate monitoring of patient's levels of oxygen and carbon dioxide. Nursing staff on general medical units typically track respiration rate and other vital signs every four hours, which is not enough. There should be a way to monitor levels continuously. Additionally, it can be hard to tell if a person's breathing rate is dangerously low in certain circumstances. There are cases, where lack of monitoring carbon dioxide level caused death.\footnote{http://abcnews.go.com/Health/parents-warn-pca-pumps-daughters-death/story?id=16796805} 

Another hazard is human mistake. For example, there is a case when nurse used a 5 mg/mL morphine cassette because a 1 mg/mL cassette was not available, but she programmed PCA Pump like for 1 mg/mL concentration. In addition to lack of monitoring of the pulse, patient died.\footnote{http://webmm.ahrq.gov/case.aspx?caseID=291}

As mentioned in chapter \ref{background}, the solution to that problem is medical devices interoperability. In addition, less human error-prone device is needed. It can be assured by using more than one system for their detection.



\section{PCA Pump Requirements Document}
\label{pcapump:requirements-doc}

Requirements of "Open Source PCA Pump" \cite{OpenSourcePCAPump:Paper} are captured in "Integrated Clinical Environment Patient-Controlled Analgesia Infusion Pump System Requirements" document \cite{PcaReq} created by Brian Larson. It is formalized set of capabilities, which Open PCA Pump should have, based on consultations with domain experts, FDA and Brian Larson's expertise gained while he was working in the medical device industry.

Conceptual model of Open PCA pump is depicted in the figure \ref{figure:ice-pca-pump}. As mentioned earlier, the pump is connected to ICE so it may be integrated with ICE apps and displays. The interface must provide prescription and patient information, current status to be displayed remotely on a supervisor user interface, and a means to stop infusing upon human command, or determination of an ICE app. Such an ICE app could monitor a patient's blood oxygenation and pulse rate, stopping the pump if depressed respiratory function is indicated. \cite{PcaReq}

\begin{figure}[ht]%t=top, b=bottom, h=here
    \begin{center}
      \includegraphics[width=0.9\textwidth]{figures/ice-pca-pump.png}      
    \end{center}
    \caption{Open PCA Pump concept}
    \label{figure:ice-pca-pump}
\end{figure}

Additionally, it cooperates with Drug Library, which contains information about drugs and its properties (like concentration). Data needed for pump operation, are captured on electronic prescription, which contains:
\begin{itemize}
    \item Patient's name
    \item Drug name
    \item Drug code
    \item Drug concentration
    \item Initial volume of drug in the vial
    \item Basal flow rate - the rate of continuous infusion
    \item Volume to be infused (VTBI) on patient's request
    \item Maximum amount of drug allowed per hour
    \item Minimum time between patient boluses
    \item Date, in which prescription has been filled
    \item Prescribing physician's name
    \item Pharmacist name
\end{itemize}

Pain medication is prescribed by a licensed physician, which is dispensed by the hospital's pharmacy. The drug is placed into a vial labeled with the name of the drug, its concentration, the prescription, and the intended patient. A clinician loads the drug into the pump, and attaches it to the patient. The pump infuses a prescribed basal flow rate which may be augmented by a patient-requested bolus or a clinician-requested bolus. This allows additional pain medication in response to patient need within safe limits. \cite{PcaReq}

Prescription captures all data needed for basal infusion and patient requested boluses (referred as bolus). In addition to that, Open PCA Pump allows Clinician Requested Bolus (refereed as square bolus). In order to do that, clinician has to enter the time (through PCA Pump panel) in which VTBI, specified in prescription, will be infused.

There can occur situations in which the maximum drug amount infused may exceed the allowed limit. E.g. when clinician issues too many square boluses. In such case, pump is switched to Keep Vein Open (KVO) mode, which has 1 ml/hr drug rate. Pump switches to KVO rate also when ICE interface request it. It may happen e.g. if patient's oxygen level is low. To recover from KVO state, pump has to be restarted by clinician in order to continue operation. In Summary, Open PCA Pump has following modes:
\begin{itemize}
    \item Stopped
    \item Basal rate
    \item Patient's bolus (bolus)
    \item Clinician bolus (square bolus)
    \item Keep Vein Open (KVO)
\end{itemize}

There are also other scenarios, which are captured by Requirements Document \cite{PcaReq}, like scanner to enable automatic entry of patient's and prescription data, occlusion detection, hardware errors alarms etc. Detailed overview of Open PCA Pump Requirements can be found in \cite{OpenSourcePCAPump:Paper}.

[MORE DETAILS ABOUT PUMP?]
[ADD STATE MACHINE IMAGE, LIKE IN UMINN REQ COD?]



\section{PCA Pump AADL/BLESS Models}
\label{pcapump:aadl-bless-models}

In addition to PCA Pump Requirements Document \cite{PcaReq}, Brian Larson created AADL model with formal behavioral specifications written in his BLESS framework. AADL model, graphical representation is depicted on figure \ref{figure:pca-pump-aadl-model}. 

\begin{figure}%[h]%t=top, b=bottom, h=here
    \begin{center}
      \includegraphics[width=0.9\textwidth]{figures/pca-pump-aadl-model.png}      
    \end{center}
    \caption{Open PCA Pump AADL model}
    \label{figure:pca-pump-aadl-model}
\end{figure}

AADL model captures structure of device. BLESS - its behavior. Listing \ref{listing:rate_controller} shows \lstinline{Rate_Controller} thread from \lstinline{PCA_Operation} component with BLESS assertions in thread declaration and BLESS behavioral description in thread implementation. The thread declaration contains input and output ports. In addition to some of them, BLESS assertions are present. Assertions are defined in BLESS annex in thread implementation. In addition to assertions, states and transitions defined in thread implementation can potentially be translated into working SPARK Ada program. Presence of timing properties in states and transitions makes translation extremely difficult, thus there are omitted in this thesis and only assertions are considered. [TRUNCATE CODE LISTING?]

\singlespacing
\begin{lstlisting}[language=aadl, frame=single, gobble=0, caption={\lstinline{Rate_Controller} thread from \lstinline{PCA_Operation} component with BLESS assertions}]
  thread Rate_Controller
  features
      Infusion_Flow_Rate: out data port PCA_Types::Flow_Rate
        {BLESS::Assertion => "<<:=PUMP_RATE()>>";};   
      System_Status: out event data port PCA_Types::Status_Type;
      Begin_Infusion: in event port
        {BLESS::Assertion => "<<Rx_APPROVED()>>";};  
      Begin_Priming: in event port;
      End_Priming: in event port;
      Halt_Infusion: in event port;
      Square_Bolus_Rate: in data port PCA_Types::Flow_Rate 
        {BLESS::Assertion => "<<:=SQUARE_BOLUS_RATE>>";};
      Patient_Bolus_Rate: in data port PCA_Types::Flow_Rate 
        {BLESS::Assertion => "<<:=PATIENT_BOLUS_RATE>>";};
      Basal_Rate: in data port PCA_Types::Flow_Rate 
        {BLESS::Assertion => "<<:=BASAL_RATE>>";};
      VTBI: in data port PCA_Types::Drug_Volume 
        {BLESS::Assertion => "<<:=VTBI>>";};
      HW_Detected_Failure: in event port;
      Stop_Pump_Completely: in event port; 
      Pump_At_KVO_Rate: in event port; 
      Alarm : in event data port PCA_Types::Alarm_Type;
      Warning : in event data port PCA_Types::Warning_Type;
      Patient_Request_Not_Too_Soon: in event port 
        {BLESS::Assertion => "<<:=PATIENT_REQUEST_NOT_TOO_SOON(now)>>";};  
      Door_Open: in data port Base_Types::Boolean;
      Pause_Infusion: in event port;
      Resume_Infusion: in event port;
      CP_Clinician_Request_Bolus: in event port;
      CP_Bolus_Duration: in event data port ICE_Types::Minute; 
      Near_Max_Drug_Per_Hour: in event port  --near maximum drug infused in any hour
        {BLESS::Assertion => "<<PATIENT_NEAR_MAX_DRUG_PER_HOUR()>>";};  
      Over_Max_Drug_Per_Hour: in event port  --over maximum drug infused in any hour
        {BLESS::Assertion => "<<PATIENT_OVER_MAX_DRUG_PER_HOUR()>>";};  
      ICE_Stop_Pump: in event port;
    properties
      Thread_Properties::Dispatch_Protocol => Aperiodic;
  end Rate_Controller;

  thread implementation Rate_Controller.imp
  annex BLESS
  {**
  assert
  --the infusion flow rate shall be:
  --  =0, after stop pump completely (from safety architecture), 
  --      clinician pressed stop button
  --  =KVO rate, after KVO rate command, or
  --      exceeded max drug per hour
  --      some alarms,
  --  =max rate, during patient-requested infusion
  --  =square bolus rate, during clinician-commanded infusion 
  --  =priming rate during pump priming
  --  =basal rate, otherwise
  <<HALT : :(la=SafetyPumpStop) or (la=StopButton) or (la=EndPriming)>>  --pump at 0 if stop button, or safety architecture says, or done priming
  <<KVO_RATE : :(la=KVOcommand) or (la=KVOalarm) or (la=TooMuchJuice)>>  --pump at KVO rate when commanded, some alarms, or excedded hourly limit
  <<PB_RATE : :la=PatientButton>>  --patient button pressed, and allowed
  <<CCB_RATE : :(la=StartSquareBolus) or (la=ResumeSquareBolus)>>  --clinician-commanded bolus start or resumption after patient bolus
  <<PRIME_RATE : :la=StartPriming>>  --priming pump
  <<BASAL_RATE : :(la=StartButton) or (la=ResumeBasal) or (la=SquareBolusDone)>>  --regular infusion
  <<PUMP_RATE : :=
    (HALT()) -> 0,                                      --no flow
    (KVO_RATE()) -> PCA_Properties::KVO_Rate,           --KVO rate
    (PB_RATE()) -> Patient_Bolus_Rate,  --maximum infusion upon patient request
    (CCB_RATE()) -> Square_Bolus_Rate,                  --square bolus rate=VTBI/duration, from data port
    (PRIME_RATE()) -> PCA_Properties::Prime_Rate,       --pump priming
    (BASAL_RATE()) -> Basal_Rate                        --basal rate, from data port
  >>
  invariant <<true>>
  variables
    --time of last action
    tla :BLESS_Types::Time := 0;
    la :   --last action
      enumeration ( 
        SafetyStopPump,   --safety architecture found a problem
        StopButton,       --clinician pressed stop button
        KVOcommand,       --from control panel (clinician) or ICE (app) to pump Keep-vein-open rate
        KVOalarm,         --some alarms should pump at KVO rate
        TooMuchJuice,     --exceeded max drug per hour, pump at KVO until prescription and patient are re-authenticated
        PatientButton,    --patient requested drug
        ResumeSquareBolus,--infusion of VTBI finished, resume clinician-commanded bolus
        ResumeBasal,      --infusion of VTBI finished, resume basal-rate
        StartSquareBolus, --begin clinician-commanded bolus
        SquareBolusDone,  --infusion of VTBI finished
        StartPriming,     --begin pump/line priming, pressed "prime" button
        EndPriming,       --end priming, pressed "prime" button again, or time-out 
        StartButton       --start pumping at basal rate
      );    
    pb_duration :BLESS_Types::Time  --patient button duration = VTBI/Patient_Bolus_Rate
      <<PB_DURATION : :pb_duration=(VTBI/Patient_Bolus_Rate)>>;
  states
    PowerOn : initial state;        --power-on
    WaitForRx : complete state;   --wait for valid prescription
    CheckPBR : state  --check Patient_Bolus_Rate is positive
      <<Rx_APPROVED()>>;
    RxApproved : complete state   --prescription verified
      <<Rx_APPROVED() and PB_DURATION()>>;
    Priming : complete state      --priming the pump, 1 ml in 6 sec
      <<(la=StartPriming) and (Infusion_Flow_Rate@now = PCA_Properties::Prime_Rate) and PB_DURATION()>>;
    WaitForStart : complete state   --wait for clinician to press 'start' button
      <<HALT() and (Infusion_Flow_Rate@now=0) and PB_DURATION()>>;
    PumpBasalRate : complete state  --pumping at basal rate
      <<((la=StartButton) or (la=ResumeBasal)) and (Infusion_Flow_Rate@now=Basal_Rate@now) and PB_DURATION()>>;
    PumpPatientButtonVTBI : complete state  --pumping patient-requested bolus
      <<(la=PatientButton) and PB_DURATION()
        and (Infusion_Flow_Rate@now=Patient_Bolus_Rate)>>;
    PumpCCBRate : complete state    --pumping at clinician-commanded bolus rate
      <<((la=StartSquareBolus) or (la=ResumeSquareBolus)) and (Infusion_Flow_Rate@now=Square_Bolus_Rate@now) and PB_DURATION()>>;
    PumpKVORate : complete state    --pumping at keep-vein-open rate
      <<((la=KVOcommand) or (la=KVOalarm) or (la=TooMuchJuice))  and PB_DURATION()
        and (Infusion_Flow_Rate@now=PCA_Properties::KVO_Rate)>>;
    PumpingSuspended : complete state  --clinician pressed 'stop' button
      <<((la=StopButton) or (la=SafetyStopPump)) and (Infusion_Flow_Rate@now=0)>>;
    Crash : final state;    --abnormal termination
    Done : final state    --normal termination
      <<Infusion_Flow_Rate@now=0>>;
  transitions
  --wait for valid prescription
    go : PowerOn-[ true ]->WaitForRx{};  
  --prescription validated
    rxo : WaitForRx-[ on dispatch Begin_Infusion ]-> CheckPBR{};
    pbr0 : CheckPBR-[ Patient_Bolus_Rate<=0 ]->Crash{}; --bad Patient_Bolus_Rate
    pbrok : CheckPBR-[ Patient_Bolus_Rate>0 ]->RxApproved
      {<<Rx_APPROVED() and (Patient_Bolus_Rate>0)>>  --likely will change from logic variable to predicate Rx_APPROVED()
        pb_duration := VTBI/Patient_Bolus_Rate  --calculate patient bolus duration
        --note division without knowing divsor is non-zero; should generate additional proof obligations for assignment using division
      <<Rx_APPROVED() and PB_DURATION()>>};  
  --clinician press 'prime' button
    rxpri : RxApproved-[ on dispatch Begin_Priming ]-> Priming  
      {
      la :=StartPriming
          <<Begin_Priming@now and Rx_APPROVED() and (la = StartPriming) and PB_DURATION()>>
      ; 
      Infusion_Flow_Rate!(PCA_Properties::Prime_Rate)  --infuse at prime rate
          <<(la = StartPriming) and Rx_APPROVED() and PB_DURATION() and
            (Infusion_Flow_Rate@now=PCA_Properties::Prime_Rate)>>   
      };
  --priming done, wait for start
    prd: Priming-[ on dispatch End_Priming or timeout (Begin_Priming) PCA_Properties::Prime_Time sec]-> WaitForStart  
      {
      la:=EndPriming
        <<HALT() and PB_DURATION()>>  --and Begin_Priming timed out
      ;
      Infusion_Flow_Rate!(0)   --stop priming flow
        <<HALT() and (Infusion_Flow_Rate@now=0) and PB_DURATION()>>
      };
  --prime again
    pri: WaitForStart-[ on dispatch Begin_Priming ]-> Priming  
      {
      la:=StartPriming
          <<Begin_Priming@now and PB_DURATION() and PRIME_RATE()>>
      ; 
      Infusion_Flow_Rate!(PCA_Properties::Prime_Rate)  --infuse at prime rate
          <<PRIME_RATE() and PB_DURATION() and
            (Infusion_Flow_Rate@now=PCA_Properties::Prime_Rate)>>   
     };
  --clinician press 'start' button after priming    
    sap: WaitForStart-[ on dispatch Begin_Infusion ]-> PumpBasalRate  --start after priming
      {
      la:=StartButton
        <<(la=StartButton) and Begin_Infusion@now  and PB_DURATION()>>    
      ;
      Infusion_Flow_Rate!(Basal_Rate)   --infuse at basal rate
        <<(la=StartButton) and (Infusion_Flow_Rate@now=Basal_Rate@now) and PB_DURATION()>>
      };
  --Patient_Request_Bolus during basal rate infusion
    pump_basal_rate: 
    PumpBasalRate-[ on dispatch Patient_Request_Not_Too_Soon]-> PumpPatientButtonVTBI
      {
      la := PatientButton 
        <<(la=PatientButton) and Patient_Request_Bolus@now and PB_DURATION()>>    
      ;
      Infusion_Flow_Rate!(Patient_Bolus_Rate)   --infuse at patient button rate
        <<(la=PatientButton) and PB_DURATION()
          and (Infusion_Flow_Rate@now=Patient_Bolus_Rate)>>     
      };  --end of pump_basal_rate
  --VTBI delivered
    vtbi_delivered: 
    PumpPatientButtonVTBI -[ on dispatch timeout (Infusion_Flow_Rate) pb_duration ms ]-> PumpBasalRate
      {
      la:=ResumeBasal
      ;
      <<(la=ResumeBasal) and PB_DURATION()>>  --and timeout of patient button duration
      Infusion_Flow_Rate!(Basal_Rate)   --infuse at basal rate
        <<(la=ResumeBasal) and (Infusion_Flow_Rate@now=Basal_Rate@now) and PB_DURATION()>>   
      };  --end of vtbi_delivered
  **};
  end Rate_Controller.imp;
\end{lstlisting} 
\label{listing:rate_controller}
\doublespacing

\section{BeagleBoard-xM}
\label{pcapump:beagleboard}

\begin{wrapfigure}{r}{0.5\textwidth}
  \begin{center}
    \includegraphics[width=0.5\textwidth]{figures/beagleboard_xm.png}
  \end{center}
  \caption{BeagleBoard-xM}
  \label{figure:beagleboard_xm}
\end{wrapfigure}

For Research and MDCF purposes, BeagleBoard-xM (an open-source hardware single-board computer produced by Texas Instruments), has been chosen as hardware platform for PCA pump prototyping.

BeagleBoard-xM is Embedded device with AM37x 1GHz ARM processor (Cortex-A8 compatible). It has 512 MB RAM, 4 USB 2.0 ports, HDMI port, 28 General-purpose input/output (GPIO) ports and Linux Operating System (on microSD card). Moreover there is PWM support, which enables control of pump actuator.

Pulse-width modulation (PWM) is a technique for controlling analog circuits with a processor's digital outputs. The average value of voltage (and current) fed to the electrical load is controlled by turning the switch between supply and load on and off at a fast pace. The longer the switch is on compared to the off periods, the higher the power supplied to the load. Proportion of on and off periods is called the duty cycle and is expressed in percent. 100\% means all the time on, 0\% - all the time off. Figure \ref{figure:pwm} shows 10\%, 30\%, 50\% and 90\% duty cycles.

\begin{figure}[ht]%t=top, b=bottom, h=here
    \begin{center}
      \includegraphics[width=0.9\textwidth]{figures/pwm.png}      
    \end{center}
    \caption{An example of PWM duty cycles}
    \label{figure:pwm}
\end{figure}

There is no existing SPARK Ada compiler running on ARM system. Hence, to compile SPARK Ada program for ARM device, cross-compiler is needed. There is GNAT compiler \cite{Horn:Thesis} created by AdaCore, but there was no cross-compiler for ARM. However, AdaCore was actively developing cross-compiler. They had working version in 2013, but tested only on their target Android-based device. In cooperation with AdaCore, cross-compiler for ARM was bundled and tested on BeagleBoard-xM. For now, GNAT cross-compiler works only on Linux 32-bit operating system.

In addition to USB ports, BeagleBoard-xM has also serial port and Ethernet port. It allows to copy programs compiled on Linux, using all three types of ports. 

%!TEX root = etdrtemplate.tex
% +--------------------------------------------------------------------+
% | Sample Chapter 4
% +--------------------------------------------------------------------+

\cleardoublepage


\chapter{AADL/BLESS to SPARK Ada translation}
\label{codegen}

First step was to create mock (based on doc, aadl models and implemented PCA Pump).
Prototyping Embedded Systems using AADL lasts for a few years \cite{PrototypyingAadl:Paper}.



\section{AADL/BLESS to SPARK Ada mapping}
\label{codegen:mapping}

%https://wiki.sei.cmu.edu/aadl/images/4/40/13_04_24-AADL-Code_Generation.pdf

%https://wiki.sei.cmu.edu/aadl/images/7/73/AADLV2Overview-AADLUserDay-Feb_2010.pdf (slide 35: port connections)

Mapping is driven by "Architecture analysis \& Design Language (AADL) V2 Programming Language Annex Document" \cite{AnnexDoc}. This document was discussed during AADL User Days in Valencia (February 2013)\footnote{http://www.aadl.info/aadl/downloads/committee/feb2013/presentations/13\_02\_04-AADL-Code\%20Generation.pdf} and in Jacksonville, FL (April 2013)\footnote{https://wiki.sei.cmu.edu/aadl/images/8/8a/Constraint\_Annex\_April22.v3.pdf}. Ocarina tool suite (based on older AADL annex documents \cite{Ocarina:Article}) and its examples\footnote{https://github.com/yoogx/polyorb-hi-ada/tree/master/examples/aadlv2} was also helpful in understanding of AADL to Ada translation.
Only high level mapping is done. No implementation (thread interactions) like Ocarina does. 


\subsection{Data types mapping}
\label{codegen:mapping:data}

[TABLE WITH TYPES MAPPING - when types mapping will be done]

During AADL/BLESS to SPARK Ada types mapping, SPARK Examiner was helpful. It detected redundancy in enumerators. Both \lstinline{Alarm_Type} and \lstinline{Warning_Type} contained \lstinline{No_Alarm} enumerators, which was a bug. \lstinline{Warning_Type} should have \lstinline{No_Warning} enumerator instead.


\subsection{AADL ports mapping}
\label{codegen:mapping:ports}

Proposed ports mapping shown in table \ref{table:aadl2spark_ports} is based on AADL runtime services from Annex 2 to "Programming Language Annex Document" \cite{AnnexDoc}. Additionaly, the mapping contains SPARK 2005 contracts.

% maybe split right column into 2 rows: spec and body?
\begin{center}
	\begin{longtable}{| p{2in} | p{4in} |}
	
		\caption{AADL to SPARK ports mapping.}
		\label{table:aadl2spark_ports}
		\\
		\hline
		\multicolumn{1}{|c|}{\textbf{AADL/BLESS}} & \multicolumn{1}{|c|}{\textbf{SPARK Ada}} \\ \hline
		\endfirsthead

		\multicolumn{2}{c}%
		{{\bfseries \tablename\ \thetable{} -- continued from previous page}} \\
		\hline 
		\multicolumn{1}{|c|}{\textbf{AADL/BLESS}} & \multicolumn{1}{|c|}{\textbf{SPARK Ada}} \\ \hline
		\endhead

		\hline \multicolumn{2}{|r|}{{Continued on next page}} \\ \hline
		\endfoot

		\hline %\hline
		\endlastfoot

		\begin{lstlisting}[language=aadl]
			Port_Name : 
				in data port Port_Type;
		\end{lstlisting} 
		&
		\begin{lstlisting}[language=ada]
			-- spec (.ads):
			procedure Receive_Port_Name;
			--# global out Port_Name;

			-- body (.adb):
			Port_Name : Port_Type;

			procedure Receive_Port_Name 
			is
			begin
				-- TODO: implement receiving Port_Name value
				-- e.g.:
				-- Port_Name := Some_Pkg.Get_Port_Name;
				null;
			end Receive_Port_Name;
		\end{lstlisting} 

		\\ \hline

		\begin{lstlisting}[language=aadl]
			Port_Name : 
				out data port Port_Type;
		\end{lstlisting} 
		&
		\begin{lstlisting}[language=ada]
			-- spec (.ads)
			function Get_Port_Name return Port_Type;
			--# global in Port_Name;

			-- body (.adb):
			Port_Name : Port_Type;

			function Get_Port_Name return Port_Type 
			is
			begin
				return Port_Name;
			end Get_Port_Name;
		\end{lstlisting} 

		\\ \hline

		\begin{lstlisting}[language=aadl]
			Port_Name : 
				in event port;
		\end{lstlisting} 
		&
		\begin{lstlisting}[language=ada]
			-- spec (.ads)
			procedure Put_Port_Name;

			-- body (.adb):
			procedure Put_Port_Name 
			is
			begin
				-- TODO: implement event handler
				null;
			end Put_Port_Name;
		\end{lstlisting} 

		\\ \hline

		\begin{lstlisting}[language=aadl]
			Port_Name : 
				out event port;
		\end{lstlisting} 
		&
		\begin{lstlisting}[language=ada]
			-- spec (.ads)
			procedure Send_Port_Name;

			-- body (.adb):

			procedure Send_Port_Name 
			is
			begin
				-- TODO: implement receiving Port_Name value
				-- e.g.:
				-- Some_Pkg.Put_Port_Name;
				null;
			end Send_Port_Name;
		\end{lstlisting} 

		\\ \hline

		\begin{lstlisting}[language=aadl]
			Port_Name : 
				in event data port Port_Type;
		\end{lstlisting} 
		&
		\begin{lstlisting}[language=ada]
			-- spec (.ads)
			procedure Put_Port_Name(Port_Name_In : Port_Type);
			--# global out Port_Name;
			--# derives Port_Name from Port_Name_In;

			-- body (.adb):
			Port_Name : Port_Type;

			procedure Put_Port_Name (Port_Name_In : Port_Type) 
			is
			begin
				Port_Name := Port_Name_In;
			end Put_Port_Name;
		\end{lstlisting} 

		\\ \hline

		\begin{lstlisting}[language=aadl]
			Port_Name : 
				out event data port Port_Type;
		\end{lstlisting} 
		&
		\begin{lstlisting}[language=ada]
			-- spec (.ads)
			procedure Send_Port_Name;
    		--# global in Port_Name;

			-- body (.adb):
			Port_Name : Port_Type;

			procedure Send_Port_Name 
			is
			begin
				-- TODO: implement receiving Port_Name value
				-- e.g.:
				-- Some_Pkg.Put_Port_Name(Port_Name);
				null;
			end Send_Port_Name;
		\end{lstlisting} 
	\end{longtable}
\end{center}


\subsection{Thread to task mapping}
\label{codegen:mapping:threads}

Thread features are mapped into SPARK Ada tasks according to table \ref{table:threads2tasks}.
There is another proposition: map threads to subprograms (table \ref{table:threads2subprograms}) in case of single threaded programs. 
BOTH TABLES ARE INCOMPLETE!

\begin{table}[!ht]
	\caption{AADL threads to SPARK Ada subprograms(procedures/functions) mapping.}
	\label{table:threads2subprograms}
	\centering
  	\begin{tabular}{ | p{3.5in} | p{2.5in} |}
	  	%\multicolumn{1}{c}{\textbf{AADL/BLESS}} & \textbf{SPARK Ada}\\

		\hline
		\multicolumn{1}{|c|}{\textbf{AADL/BLESS}} & \multicolumn{1}{|c|}{\textbf{SPARK Ada}} \\ \hline

		\begin{lstlisting}[language=aadl]
			package Some_Pkg
				thread Some_Thread
					features
						Some_Port : out data port Port_Type;
				end Some_Thread;
			end Some_Pkg;
		\end{lstlisting} 
		& 
		\begin{lstlisting}
			package Some_Pkg.Some_Thread
			is
				function Get_Some_Port return Port_Type;
			end Some_Pkg.Some_Thread;
		\end{lstlisting} 

		\\ \hline

		\begin{lstlisting}[language=aadl]
			package Some_Pkg
				thread Some_Thread
					features
						Some_Port : out data port Port_Type
							{Compute_Entrypoint_Source_Text 
								=> "Custom_Pkg.Custom_Subprogram";};
				end Some_Thread;
			end Some_Pkg;
		\end{lstlisting} 
		& 
		\begin{lstlisting}
			package Custom_Pkg
			is
				function Custom_Subprogram return Port_Type;
			end Custom_Pkg;

		\end{lstlisting} 

		\\ \hline
	\end{tabular}
\end{table}

\begin{table}[!ht]
	\caption{AADL threads to SPARK Ada tasks mapping.}
	\label{table:threads2tasks}
	\centering
  	\begin{tabular}{ | p{3.5in} | p{2.5in} |}
	  	%\multicolumn{1}{c}{\textbf{AADL/BLESS}} & \textbf{SPARK Ada}\\

		\hline
		\multicolumn{1}{|c|}{\textbf{AADL/BLESS}} & \multicolumn{1}{|c|}{\textbf{SPARK Ada}} \\ \hline

		\begin{lstlisting}[language=aadl]
			package Some_Pkg
				thread Some_Thread
					features
						Some_Port : out data port Port_Type;
				end Some_Thread;
			end Some_Pkg;
		\end{lstlisting} 
		& 
		\begin{lstlisting}
			package Some_Pkg.Some_Thread
			is
				task type Some_Thread
				--# global out Some_Port;
				is
					pragma Priority(10);
				end Some_Thread;
			end Some_Pkg.Some_Thread;
		\end{lstlisting} 

		\\ \hline

		\begin{lstlisting}[language=aadl]
			package Some_Pkg
				thread Some_Thread.imp
				end Some_Thread;
			end Some_Pkg;
		\end{lstlisting} 
		& 
		\begin{lstlisting}
			st : Some_Thread;
			package body Custom_Pkg
			is
				task body Some_Thread
				is
				begin
					null;
				end Some_Thread;
			end Custom_Pkg;
		\end{lstlisting} 

		\\ \hline
	\end{tabular}
\end{table}

AADL package, which contains threads is split into child packages with convention: AADL\_Package\_Name -> AADL\_Package\_Name.Thread\_Name.

In SPARK Ada there are nested packages and child packages. Sample nested packages are shown in listing \ref{lst:nested_packages}. Equivalent child packages are shown in listing \ref{lst:child_packages}. The name of a child package consists of the parent unit's name followed by the child package's identifier, separated by a period (dot) `.'. Calling convention is the same for child and nested packages (e.g. \lstinline{P.N} in listings \ref{lst:nested_packages} and \ref{lst:child_packages}. However, there is a difference between nested packages and child packages. In nested package declarations become visible as they are introduced, in textual order. For example, in listing \ref{lst:nested_packages} spec \lstinline{N} cannot refer to \lstinline{M} in any way. In case of child packages, with certain exceptions, all the functionality of the parent is available to a child and parent can access all its child packages. More precisely: all public and private declarations of the parent package are visible to all child packages. Private child package can be accessed only from parent's body.

\begin{lstlisting}[language=ada, frame=single, gobble=0, caption={Nested packages in SPARK Ada}, label={lst:nested_packages}]
	package P is
	   D: Integer;

	   --  a nested package:
	   package N is
	      X: Integer;
	   private
	      Foo: Integer;
	   end N;

	   E: Integer;
	private
	   --  nested package in private section:
	   package M is
	      Y: Integer;
	   private
	      Bar: Integer;
	   end M;

	end P;
\end{lstlisting}

\begin{lstlisting}[language=ada, frame=single, gobble=0, caption={Child packages in SPARK Ada}, label={lst:child_packages}]
	package P is
	   D: Integer;
	   E: Integer;
	end P;

	--  a child package:
	package P.N is
      X: Integer;
   	private
      Foo: Integer;
	end P.N;

	--  a child private package:
	private package M is
	  Y: Integer;
	private
	  Bar: Integer;
	end M;
\end{lstlisting}


\subsection{Subprograms mapping}
\label{codegen:mapping:subprograms}

I added Subprograms to existing PCA Pump AADL models (???) etc.
How I did it. Code examples.

\begin{table}[!ht]
	\caption{AADL subprograms to SPARK Ada subprograms(procedures/functions) mapping.}
	\label{table:subprograms_mapping}
	\centering
  	\begin{tabular}{ | p{3in} | p{3in} |}
	  	%\multicolumn{1}{c}{\textbf{AADL/BLESS}} & \textbf{SPARK Ada}\\

		\hline
		\multicolumn{1}{|c|}{\textbf{AADL/BLESS}} & \multicolumn{1}{|c|}{\textbf{SPARK Ada}} \\ \hline

		\begin{lstlisting}[language=aadl]
			subprogram sp
			features
				e : in parameter T;
				s : out parameter T;
			end sp;
		\end{lstlisting} 
		& 
		\begin{lstlisting}
			procedure sp(e : in T; s : out T) is 
			begin
				null;
			end sp;
		\end{lstlisting} 

		\\ \hline

		\begin{lstlisting}[language=aadl]
			data Flow_Rate  --dose rate
  				properties
    				BLESS::Typed=>"integer";
    				Data_Model::Base_Type => (classifier(Base_Types::Integer_16));
    				Data_Model::Measurement_Unit => "ml/hr";
			end Flow_Rate;
		\end{lstlisting} 
		& 
		\begin{lstlisting}
			subtype Flow_Rate is Integer range 0 .. Integer'Last;
		\end{lstlisting} 

		\\ \hline
	\end{tabular}
\end{table}


\subsection{Ports communication mapping}

This is a problem: "consumer.ads:1:13: Semantic Error 135 - The package Producer is undeclared or not visible, or there is a circularity in the list of inherited packages.".


\subsection{BLESS mapping}
\label{codegen:mapping:bless}

\begin{center}
	\begin{longtable}{| p{3in} | p{3in} |}
		\caption{BLESS to SPARK contracts mapping.}
		\label{table:bless2spark}
		\\
		\hline
		\multicolumn{1}{|c|}{\textbf{AADL/BLESS}} & \multicolumn{1}{|c|}{\textbf{SPARK Ada}} \\ \hline
		\endfirsthead

		\multicolumn{2}{c}%
		{{\bfseries \tablename\ \thetable{} -- continued from previous page}} \\
		\hline 
		\multicolumn{1}{|c|}{\textbf{AADL/BLESS}} & \multicolumn{1}{|c|}{\textbf{SPARK Ada}} \\ \hline
		\endhead

		\hline \multicolumn{2}{|r|}{{Continued on next page}} \\ \hline
		\endfoot

		\hline %\hline
		\endlastfoot

		\begin{lstlisting}[language=bless]
			BLESS::Assertion=>"<<COND1()>>"
		\end{lstlisting} 
		& 
		\begin{lstlisting}
			--# assert COND1();
		\end{lstlisting} 

		\\ \hline

		\begin{lstlisting}[language=bless]
			thread Some_Thread
			features
				Some_Port : out event port
				{BLESS:Assertion => "<<(Var1 < Var2 and COND2())>>";};
			end Some_Thread;
		\end{lstlisting} 
		& 
		\begin{lstlisting}
			task body Some_Thread
			is
			begin
				loop
					--# assert (Var1 < Var2 and COND2());
				end loop;
			end Some_Thread;
		\end{lstlisting} 

		\\ \hline

		\begin{lstlisting}[language=bless]
			thread implementation Some_Thread.imp
			annex BLESS 
			{**
				invariant <<(Some_Var < Other_Var)>>
			**};
			end Some_Thread.imp;
		\end{lstlisting} 
		& 
		\begin{lstlisting}
			task body Some_Thread
			is
			begin
				loop
					--# assert (Some_Var < Other_Var);
				end loop;
			end Some_Thread;
		\end{lstlisting} 

		\\ \hline

		\begin{lstlisting}[language=bless]
			thread implementation Some_Thread.imp
			annex BLESS 
			{**
				assert
				<<State1 : :COND1() or COND2()>>
				<<Var : :=
  								(State1()) -> 0,
  								(State2()) -> -1,
  								(State3()) -> 9
				>>
			**};
			end Some_Thread.imp;
		\end{lstlisting} 
		& 
		\begin{lstlisting}
			task body Some_Thread
			is
			begin
				loop
					--# assert COND1() or COND2() 
					--#          -> State1();
					--# assert (Var = 0) -> State1() and
					--#        (Var = -1) -> State2() and
					--#        (Var = 9) -> State3();
				end loop;
			end Some_Thread;
		\end{lstlisting} 

		\\ \hline

		\begin{lstlisting}[language=bless]
			subprogram Some_Subprogram
			features 
				param : out parameter Base_Types::Integer;
			annex subBless
			{**
				pre <<(param > 0)>>
				post <<(param = 0)>>
			**};
			end Some_Subprogram;
		\end{lstlisting} 
		& 
		\begin{lstlisting}
			procedure Some_Subprogram(Param : in out Integer);
		    --# pre Param > 0;
		    --# post Param = 0;
		\end{lstlisting} 

		\\ \hline

		\begin{lstlisting}[language=bless]
			<<Pre()>>Action()<<Post()>>
		\end{lstlisting} 
		& 
		\begin{lstlisting}
			procedure Action;
			--# pre Pre;
			--# post Post;
		\end{lstlisting} 

		\\ \hline
		

		\begin{lstlisting}[language=bless]
			<<Pre()>>Action()<<Post()>>
		\end{lstlisting} 
		& 
		\begin{lstlisting}
			procedure Action;
			--# pre Pre;
			--# post Post;
		\end{lstlisting} 

		\\ \hline
	\end{longtable}
\end{center}

Generated (translated) code will not be complete. It will still require Developer's effort to implement missing parts. E.g. when assertion is not defined, it is developer responsibility to implement it.


\section{"DeusEx" translator}
\label{codegen:translator}

The ultimate goal is to perform, translation described in \ref{codegen:mapping} automatically. "DeusEx" translator will enable to perform translation of entire model and parts of the model. Initially, following functions will be supported:
\begin{itemize}
	\item types translation
	\item threads to tasks translation
	\item subprogram to procedure/function translation
	\item single package translation
\end{itemize}

Translator will be created in Scala programming language.


%!TEX root = etdrtemplate.tex
% +--------------------------------------------------------------------+
% | Sample Chapter N-1
% +--------------------------------------------------------------------+

\cleardoublepage

% +--------------------------------------------------------------------+
% | Replace "This is Chapter N-1" below with the title of your chapter.
% | LaTeX will automatically number the chapters.
% +--------------------------------------------------------------------+

\chapter{Summary}
\label{summary}

What I have done.

%!TEX root = JakubJedryszek-MasterThesis.tex

\cleardoublepage

\chapter{Future work}
\label{future_work}

The following are possible extensions for work done in this thesis:

\begin{itemize}
	\item The most important thing, which would be extremely helpful to proceed with work done in this thesis, would be to review it by some industry expert and experienced engineer.
	\item Creation of automatic translator described in section \ref{codegen:translator} would be good validation of created translation schemes. It may reveal some issues not present for manual translation.
	\item Currently AADL thread properties are not take into account in thread to task mapping, in section \ref{codegen:mapping:threads}. Properties like priority or period would be very useful in SPARK Ada programs. For now, former is hard-coded as 10 and latter simply skipped, which requires developer to handle it. However, such property modeled and analyzed in AADL models, should be translated automatically to maintain synchronization between model and the code. AADL properties are described in \cite{AadlBook}, in the Appendix A.
	\item Data types translation presented in section \ref{codegen:mapping:data}, in additions to straightforward type mapping, comprises of protected types. However, all protected types has the same set of subprograms (\lstinline{Put} and \lstinline{Get}). It is worth to consider introduction of generics, which will allow to specify generic protected type and then reuse it for all types.
	\item In feature groups translation (section \ref{codegen:mapping:feature_groups}), idea of child or nested packages is abandoned. However, it would be good to reconsider it. Maybe by introduction of getter functions in parent package or some other techniques, which will allow for better separation and decomposition.
	\item AADL property set mapping in section \ref{codegen:mapping:propertyset} handles only \lstinline{aadlinteger} type. Thus, it requires extension for handling other, more complex constructs.
	\item Current translation schemes cause creation of pretty big packages, which will become bigger after adding implementation. Thus, some decomposition is desired. Following techniques can be considered: 
		\begin{itemize}
			\item partition of packages
			\item take advantage of child packages
			\item separation of threads to different packages (e.g. one thread per child package and all common functionalities in parent package)
		\end{itemize} simple package separation
	\item Mapping for BLESS are limited only to small subset. Development of translations for BLESS state machine (states and transitions) would be good addition. It will allow for behavior translation. Good point to start is \lstinline{Rate_Controller} thread, which can be found in \lstinline{PCA_Operation_Threads} package in original AADL models created by Brian Larson. The semantics of BLESS contain notions of time that make translation to SPARK difficult. This problem occurs in state machine models. Finding solution for that is needed. Maybe even, by changing BLESS semantics.
	\item For the time, when this thesis was written, SPARK 2014 did not support multitasking. However, there were plans to introduce it into SPARK 2014 like it took place in case of SPARK 2005. Once, multitasking support would be present, translations for SPARK 2014 will be possible.
	\item There is an issue with two way communication between SPARK packages caused by circular dependency. It is described in section \ref{codegen:port_communication:thread}.
	\item Port communication presented in section \ref{codegen:port_communication} captures only 1:1 connections between ports of the same type and opposite direction. In AADL there are also inter-port connections and one-to-many or many-to-one connections. \cite{AadlBook} They should be taken into AADL subset for medical devices modeling and translation. 
	\item Created PCA pump prototype contains only basic functionalities. Some parameters (like drug concentration) are ignored. The next step is its development, would be taken skipped parts into account. In addition to that, interaction with external modules, like sensors for monitoring drug flow, or communication with ICE through Ethernet port is desired. It requires creation of communication channel 	between BeagleBoard (SPARK Ada application) and these systems. 		
\end{itemize}


% +-------------------------------------------------------------------------+
% | References                                                              |
% +-------------------------------------------------------------------------+

% +-------------------------------------------------------------------------+
% | In order for WinEDT to index references correctly, it has to know where |
% | the file resides.  The following command is prefaced by %, and will be  |
% | ignored completely by LaTeX.  However, WinEDT will use this line to     |
% | access the external .bib bibliography file.  Also note that WinEDT can  |
% | read file path names with either "\" or "/" - LaTeX, however, doesn't   |
% | like "\", so it's easier to store a path name in the "Unix" style.      |
% +-------------------------------------------------------------------------+

%Included for Gather Purpose only.  Do NOT uncomment:
%input "references.bib"

% +--------------------------------------------------------------------+
% | This template uses the BibTeX program to format references.  The
% | lines below create a separate Bibliography section and add
% | an entry for "Bibliography" to the Table of Contents.  The actual
% | data for your references (author, title, journal, date, etc.) are
% | entered in the references.bib file.  See that file for information
% | on how to enter references.
% +--------------------------------------------------------------------+

\cleardoublepage
\phantomsection
\addcontentsline{toc}{chapter}{Bibliography}
\bibdata{references}
\bibliography{references}

% +--------------------------------------------------------------------+
% | Finally, we generate the appendix.  To add or delete appendices,
% | add or remove the line
% |
% |     \input{appendixX.tex}
% |
% | where "X" is the letter designation of the Appendix (A, B, C, etc.)
% | You should have one \input{appendixX.tex} line and a corresponding
% | file appendixX.tex for each appendix.                                 |
% +--------------------------------------------------------------------+

\appendix
%!TEX root = etdrtemplate.tex
% +--------------------------------------------------------------------+
% | Appendix A Page (Optional)                                         |
% +--------------------------------------------------------------------+

\cleardoublepage

\chapter{Title for This Appendix}
\label{Appendix:Key1}

Content of this appendix.


% +--------------------------------------------------------------------+
% | Enter text for your Appendix page in the space below this box.     |
% |                                                                    |
% +--------------------------------------------------------------------+

%!TEX root = JakubJedryszek-MasterThesis.tex

\cleardoublepage

\chapter{PCA Pump Prototype - translated from AADL/BLESS}
\label{Appendix:PPP_Trahslated}

%code listings with numbers


\end{document}
