% +--------------------------------------------------------------------+
% | Acknowledgements Page (Optional)                                   |
% +--------------------------------------------------------------------+

\newpage
\vspace*{0.9cm}
\begin{center}
{\bf \Huge Acknowledgments}
\end{center}

\setlength{\baselineskip}{0.8cm}

%\pdfbookmark[0]{Acknowledgements}{PDF_Acknowledgements}

% +--------------------------------------------------------------------+
% | Enter text for your acknowledgements in the space below this box.  |
% |                                                                    |
% +--------------------------------------------------------------------+

This template uses a separate file for each section of your ETDR:
title page, abstract, preface, chapters, reference, etc.  This
makes it easier to organize and work with a lengthy document.  The
template is configured with page margins required by the Graduate
School and will automatically create a table of contents, lists of
tables and figures, and PDF bookmarks.

Although the template gives you a foundation for creating your
ETDR, you will need a working knowledge of LaTeX in order to
produce a final document.  You should be familiar with LaTeX
commands for formatting text, equations, tables, and other
elements you will need to include in your ETDR.

This template uses a separate file for each section of your ETDR:
title page, abstract, preface, chapters, reference, etc.  This
makes it easier to organize and work with a lengthy document.  The
template is configured with page margins required by the Graduate
School and will automatically create a table of contents, lists of
tables and figures, and PDF bookmarks.

Although the template gives you a foundation for creating your
ETDR, you will need a working knowledge of LaTeX in order to
produce a final document.  You should be familiar with LaTeX
commands for formatting text, equations, tables, and other
elements you will need to include in your ETDR.

This template uses a separate file for each section of your ETDR:
title page, abstract, preface, chapters, reference, etc.  This
makes it easier to organize and work with a lengthy document.  The
template is configured with page margins required by the Graduate
School and will automatically create a table of contents, lists of
tables and figures, and PDF bookmarks.

Although the template gives you a foundation for creating your
ETDR, you will need a working knowledge of LaTeX in order to
produce a final document.  You should be familiar with LaTeX
commands for formatting text, equations, tables, and other
elements you will need to include in your ETDR.
