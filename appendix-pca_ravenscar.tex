%!TEX root = JakubJedryszek2014.tex

\cleardoublepage

\chapter{PCA pump prototype - simple, implemented, working pump}
\label{Appendix:pca_ravenscar}

This appendix contains implemented, simple version of PCA Pump, created based on \ref{pcapump:requirements-doc} and AADL models created by Brian Larson. Data types used by this pump are the same like translated from AADL models presented in appendix \ref{Appendix:pca_generated}.

\singlespacing
\begin{lstlisting}[language=ada, gobble=0, numbers=left, caption={\lstinline{Pca_Operation} package}, label={listing:pca_ravenscar:pca_operation}]
with Base_Types;
with Pca_Types;
with Ice_Types;
--# inherit Ada.Real_Time,
--#         Ada.Synchronous_Task_Control,
--#         Base_Types,
--#         Pca_Types,
--#         Ice_Types,
--#         Pca_Engine;
package Pca_Operation
--# own protected Operate (suspendable);
--#     protected Fluid_Pulses : Integer_Array_Store(Priority => 10);
--#     protected Prescription : Pca_Types.Prescription_Store(Priority => 10);
--#     protected State : Pca_Types.Status_Type_Store(Priority => 10);
--#     protected Clinician_Bolus_Paused : Base_Types.Boolean_Store(Priority => 10);
--#     protected Clinician_Bolus_Duration : Ice_Types.Minute_Store(Priority => 10);
--#     task rc : Rate_Controller;
--#     task mdphw : Max_Drug_Per_Hour_Watcher;
--#     Last_Patient_Bolus;
--# initializes Last_Patient_Bolus;
is

    subtype Integer_Array_Index is Integer range 1 .. 60*60;
    type Integer_Array is array (Integer_Array_Index) of Integer;

    protected type Integer_Array_Store
    is
        pragma Priority (10);

        function Get(Ind : in Integer) return Integer;
        --# global in Integer_Array_Store;

        procedure Put(Ind : in Integer; Val : in Integer);
        --# global in out Integer_Array_Store;
        --# derives Integer_Array_Store from Integer_Array_Store, Ind, Val;

        procedure Inc(Ind : in Integer);
        --# global in out Integer_Array_Store;
        --# derives Integer_Array_Store from Integer_Array_Store, Ind;

        function Sum return Integer;
        --# global in Integer_Array_Store;

        procedure Pulse;
        --# global in out Integer_Array_Store;
        --# derives Integer_Array_Store from Integer_Array_Store;
    private
        TheStoredData : Integer_Array := Integer_Array'(others => 0);
    end Integer_Array_Store;

    function Get_Volume_Infused return Integer; -- microliters
    --# global in     Fluid_Pulses;

    function Get_State return Pca_Types.Status_Type;
    --# global in     State;

    procedure Panel_Set_Basal_Flow_Rate(Flow_Rate : Pca_Types.Flow_Rate);
    --# global in out Prescription;

    function Panel_Get_Basal_Flow_Rate return Pca_Types.Flow_Rate;
    --# global in     Prescription;

    procedure Panel_Set_Vtbi(Vtbi : Pca_Types.Drug_Volume);
    --# global in out Prescription;

    function Panel_Get_Vtbi return Pca_Types.Drug_Volume;
    --# global in     Prescription;

    procedure Panel_Set_Max_Drug_Per_Hour(Max_Drug_Per_Hour : Pca_Types.Drug_Volume);
    --# global in out Prescription;

    function Panel_Get_Max_Drug_Per_Hour return Pca_Types.Drug_Volume;
    --# global in     Prescription;

    procedure Panel_Set_Minimum_Time_Between_Bolus(Minimum_Time_Between_Bolus : Ice_Types.Minute);
    --# global in out Prescription;

    function Panel_Get_Minimum_Time_Between_Bolus return Ice_Types.Minute;
    --# global in     Prescription;

    procedure StartPump;
    --# global out    Operate;
    --#        out    State;
    --# derives Operate from &
    --#         State from ;

    procedure StopPump;
    --# global out    Operate;
    --#        out    State;
    --# derives Operate from &
    --#         State from ;

    procedure PatientBolus;
    --# global in out State;
    --#        in out Last_Patient_Bolus;
    --#        in out Clinician_Bolus_Paused;
    --#        in     Prescription;
    --#        in     Ada.Real_Time.ClockTime;
    --# derives State from State, Last_Patient_Bolus, Prescription, Ada.Real_Time.ClockTime &
    --#         Last_Patient_Bolus from Last_Patient_Bolus, Prescription, Ada.Real_Time.ClockTime &
    --#         Clinician_Bolus_Paused from Clinician_Bolus_Paused, State, Last_Patient_Bolus, Prescription, Ada.Real_Time.ClockTime;

    procedure ClinicianBolus(Cb_Duration : in     Ice_Types.Minute);
    --# global in out State;
    --#        in out Clinician_Bolus_Duration;
    --#        in out Clinician_Bolus_Paused;
    --# pre Cb_Duration <= 6 * 60; -- from Requirements 4.3.5


private
    task type Rate_Controller
    --# global out Operate;
    --#        in out Fluid_Pulses;
    --#        in out Clinician_Bolus_Paused;
    --#        in     Prescription;
    --#        in out State;
    --#        in     Ada.Real_Time.ClockTime;
    --#        in     Clinician_Bolus_Duration;
    --# derives Operate       from &
    --#         Fluid_Pulses from Fluid_Pulses, State, Clinician_Bolus_Paused, Ada.Real_Time.ClockTime, Prescription, Clinician_Bolus_Duration &
    --#         Clinician_Bolus_Paused from State, Clinician_Bolus_Paused, Ada.Real_Time.ClockTime, Prescription, Clinician_Bolus_Duration &
    --#         State        from State, Prescription, Ada.Real_Time.ClockTime, Clinician_Bolus_Duration, Clinician_Bolus_Paused;
    --# declare suspends => Operate;
    is
        pragma Priority (9);
    end Rate_Controller;

    task type Max_Drug_Per_Hour_Watcher
    --# global in out Fluid_Pulses;
    --#        in     Prescription;
    --#        in out State;
    --#        in     Ada.Real_Time.ClockTime;
    --# derives Fluid_Pulses  from Fluid_Pulses &
    --#         State         from Prescription, Fluid_Pulses, State &
    --#         null          from Ada.Real_Time.ClockTime;
    is
        pragma Priority (9);
    end Max_Drug_Per_Hour_Watcher;

end Pca_Operation;

with Ada.Synchronous_Task_Control,
     Ada.Real_Time,
     Base_Types,
     Pca_Types,
     Pca_Engine;
use type Ada.Real_Time.Time;
use type Pca_Types.Status_Type;

package body Pca_Operation
is
    Operate : Ada.Synchronous_Task_Control.Suspension_Object;
    rc : Rate_Controller;
    mdphw : Max_Drug_Per_Hour_Watcher;

    Fluid_Pulses : Integer_Array_Store;

    State : Pca_Types.Status_Type_Store;

    Prescription : Pca_Types.Prescription_Store;

    Fluid_Pulse_Volume : constant Natural := 100; -- in microliters

    Kvo_Rate : constant Pca_Types.Flow_Rate := 1; -- in mililiters

    Bolus_Flow_Rate : constant Pca_Types.Flow_Rate := 100; -- in mililiters

    Last_Patient_Bolus : Ada.Real_Time.Time := Ada.Real_Time.Time_First;

    Clinician_Bolus_Duration : Ice_Types.Minute_Store;

    Clinician_Bolus_Paused : Base_Types.Boolean_Store;

    protected body Integer_Array_Store
    is
        function Get(Ind : in Integer) return Integer
        --# global in TheStoredData;
        is
        begin
            return TheStoredData(Ind);
        end Get;


        procedure Put(Ind : in Integer; Val : in Integer)
          --# global in out TheStoredData;
          --# derives TheStoredData from TheStoredData, Ind, Val;
        is
        begin
            TheStoredData(Ind) := Val;
        end Put;

        procedure Inc(Ind : in Integer)
          --# global in out TheStoredData;
          --# derives TheStoredData from TheStoredData, Ind;
        is
        begin
            TheStoredData(Ind) := TheStoredData(Ind) + Fluid_Pulse_Volume;
        end Inc;

        function Sum return Integer
        --# global in TheStoredData;
        is
            Result : Integer := 0;
        begin
            for I in Integer_Array_Index loop
                --# assert I > 1 -> Result >= TheStoredData(I-1);
                Result := Result + TheStoredData(I);
            end loop;
            return Result;
        end Sum;

        procedure Pulse
        --# global in out TheStoredData;
        --# derives TheStoredData from TheStoredData;
        is
        begin
            for I in Integer_Array_Index range 1 .. Integer_Array_Index'Last-1 loop
                --# assert I > 1 -> TheStoredData(I-1) = TheStoredData(I);
                TheStoredData(I) := TheStoredData(I+1);
            end loop;
            TheStoredData(Integer_Array_Index'Last) := 0;
        end Pulse;

    end Integer_Array_Store;

    function Get_Time_Between_Activations(Flow_Rate : in Pca_Types.Flow_Rate) return Natural
    is
        Result : Natural;
        Flow_Rate_In_Microliters : Natural;
        Activations_Per_Hour : Natural;
    begin
        Flow_Rate_In_Microliters := Natural(Flow_Rate) * 1000; -- convert mL to uL
        Activations_Per_Hour := Flow_Rate_In_Microliters / Fluid_Pulse_Volume;

        -- miliseconds between activations
        Result := ((60 * 60) * 1000) / Activations_Per_Hour;
        return Result;
    end Get_Time_Between_Activations;


    function Get_Volume_Infused return Integer
    is
    begin
        return Fluid_Pulses.Sum;
    end Get_Volume_Infused;

    function Get_State return Pca_Types.Status_Type
    is
        Current_State : Pca_Types.Status_Type;
    begin
        Current_State := State.Get;
        return Current_State;
    end Get_State;

    procedure Panel_Set_Basal_Flow_Rate(Flow_Rate : Pca_Types.Flow_Rate)
    is
    begin
        Prescription.Set_Basal_Flow_Rate(Flow_Rate);
    end Panel_Set_Basal_Flow_Rate;

    function Panel_Get_Basal_Flow_Rate return Pca_Types.Flow_Rate
    is
    begin
        return Prescription.Get_Basal_Flow_Rate;
    end Panel_Get_Basal_Flow_Rate;

    procedure Panel_Set_Vtbi(Vtbi : Pca_Types.Drug_Volume)
    is
    begin
        Prescription.Set_Vtbi(Vtbi);
    end Panel_Set_Vtbi;

    function Panel_Get_Vtbi return Pca_Types.Drug_Volume
    is
    begin
        return Prescription.Get_Vtbi;
    end Panel_Get_Vtbi;

    procedure Panel_Set_Max_Drug_Per_Hour(Max_Drug_Per_Hour : Pca_Types.Drug_Volume)
    is
    begin
        Prescription.Set_Max_Drug_Per_Hour(Max_Drug_Per_Hour);
    end Panel_Set_Max_Drug_Per_Hour;

    function Panel_Get_Max_Drug_Per_Hour return Pca_Types.Drug_Volume
    is
    begin
        return Prescription.Get_Max_Drug_Per_Hour;
    end Panel_Get_Max_Drug_Per_Hour;

    procedure Panel_Set_Minimum_Time_Between_Bolus(Minimum_Time_Between_Bolus : Ice_Types.Minute)
    is
    begin
        Prescription.Set_Minimum_Time_Between_Bolus(Minimum_Time_Between_Bolus);
    end Panel_Set_Minimum_Time_Between_Bolus;

    function Panel_Get_Minimum_Time_Between_Bolus return Ice_Types.Minute
    is
    begin
        return Prescription.Get_Minimum_Time_Between_Bolus;
    end Panel_Get_Minimum_Time_Between_Bolus;

    procedure StartPump
    is
    begin
        Ada.Synchronous_Task_Control.Set_True (Operate);
        State.Put(Pca_Types.Basal);
    end StartPump;

    procedure StopPump
    is
    begin
        Ada.Synchronous_Task_Control.Set_False (Operate);
        State.put(Pca_Types.Stopped);
    end StopPump;

    procedure PatientBolus
    is
        Minimum_Time_Between_Bolus : Ice_Types.Minute;
        Time_Now : Ada.Real_Time.Time;
        Current_State : Pca_Types.Status_Type;
    begin
        Minimum_Time_Between_Bolus := Prescription.Get_Minimum_Time_Between_Bolus;
        Time_Now := Ada.Real_Time.Clock;
        if Last_Patient_Bolus + Ada.Real_Time.Milliseconds(Natural(Minimum_Time_Between_Bolus)*(60*1000)) <= Time_Now then
            Last_Patient_Bolus := Time_Now;
            Current_State := State.Get;
            if Current_State = Pca_Types.Square_Bolus then
                Clinician_Bolus_Paused.Put(True);
            end if;
            State.Put(Pca_Types.Bolus);
        end if;
    end PatientBolus;

    procedure ClinicianBolus(Cb_Duration : in     Ice_Types.Minute)
    is
        Current_State : Pca_Types.Status_Type;
    begin
        Current_State := State.Get;
        if Current_State = Pca_Types.Basal then
            Clinician_Bolus_Duration.Put(Cb_Duration);
            State.Put(Pca_Types.Square_Bolus);
        elsif Current_State = Pca_Types.Bolus then
            Clinician_Bolus_Duration.Put(Cb_Duration);
            Clinician_Bolus_Paused.Put(True);
        end if;
    end ClinicianBolus;


    task body Rate_Controller
    is
        --Release_Time : Ada.Real_Time.Time;
        Now : Ada.Real_Time.Time;
        Period : Natural;
        Last_Basal_Pulse : Ada.Real_Time.Time := Ada.Real_Time.Time_First; -- Ada.Real_Time.Clock - Ada.Real_Time.Milliseconds(1000 * 60 * 60);
        Last_Kvo_Pulse : Ada.Real_Time.Time := Ada.Real_Time.Time_First; -- Ada.Real_Time.Clock - Ada.Real_Time.Milliseconds(1000 * 60 * 60);
        Last_Patient_Bolus_Pulse : Ada.Real_Time.Time := Ada.Real_Time.Time_First; -- Ada.Real_Time.Clock - Ada.Real_Time.Milliseconds(1000 * 60 * 60);
        Last_Clinician_Bolus_Pulse : Ada.Real_Time.Time := Ada.Real_Time.Time_First; -- Ada.Real_Time.Clock - Ada.Real_Time.Milliseconds(1000 * 60 * 60);
        Patient_Bolus_Volume_Left : Natural := 0;
        Clinicaian_Bolus_Vtbi : Pca_Types.Drug_Volume;
        Clinician_Bolus_Volume_Left : Natural := 0;
        Current_State : Pca_Types.Status_Type;
        Flow_Rate : Pca_Types.Flow_Rate;
        Drug_Volume : Pca_Types.Drug_Volume;
        Clinician_Bolus_Paused_Temp : Boolean;
        Clinician_Bolus_Duration_Temp : Ice_Types.Minute;
    begin
        loop
            Ada.Synchronous_Task_Control.Suspend_Until_True (Operate); -- wait until user allows Pump to operate
            Ada.Synchronous_Task_Control.Set_True (Operate); -- Keep the task running, the previous call will have set Operate to False.

            Now := Ada.Real_Time.Clock;
            Current_State := State.Get;

            --# assert true;

            case Current_State is
            when Pca_Types.Stopped =>
                null;
            when Pca_Types.KVO =>
                Period := Get_Time_Between_Activations(Kvo_Rate);
                if Last_Kvo_Pulse + Ada.Real_Time.Milliseconds(Period) <= Now then
                    Last_Kvo_Pulse := Now;
                    Fluid_Pulses.Inc(Integer_Array_Index'Last); -- each time round, update the volume infused
                    Pca_Engine.Run_Pumping(100); -- and pump 0.1 ml
                end if;
            when Pca_Types.Basal =>
                Flow_Rate := Prescription.Get_Basal_Flow_Rate;
                Period := Get_Time_Between_Activations(Flow_Rate);
                if Last_Basal_Pulse + Ada.Real_Time.Milliseconds(Period) <= Now then
                    Last_Basal_Pulse := Now;
                    Fluid_Pulses.Inc(Integer_Array_Index'Last); -- each time round, update the volume infused
                    Pca_Engine.Run_Pumping(100); -- and pump 0.1 ml
                end if;
            when Pca_Types.Bolus =>
                -- basal
                Flow_Rate := Prescription.Get_Basal_Flow_Rate;
                Period := Get_Time_Between_Activations(Flow_Rate);
                if Last_Basal_Pulse + Ada.Real_Time.Milliseconds(Period) <= Now then
                    Last_Basal_Pulse := Now;
                    Fluid_Pulses.Inc(Integer_Array_Index'Last); -- each time round, update the volume infused
                    Pca_Engine.Run_Pumping(100); -- and pump 0.1 ml
                end if;

                -- patient
                Period := Get_Time_Between_Activations(Bolus_Flow_Rate);
                if Last_Patient_Bolus_Pulse + Ada.Real_Time.Milliseconds(Period) <= Now then
                    Last_Patient_Bolus_Pulse := Now;

                    if Patient_Bolus_Volume_Left = 0 then
                        Drug_Volume := Prescription.Get_Vtbi;
                        Patient_Bolus_Volume_Left := Natural(Drug_Volume);
                        Patient_Bolus_Volume_Left := Patient_Bolus_Volume_Left * 1000; -- convert to microliters
                    end if;

                    Fluid_Pulses.Inc(Integer_Array_Index'Last); -- each time round, update the volume infused
                    Pca_Engine.Run_Pumping(100); -- and pump 0.1 ml

                    Patient_Bolus_Volume_Left := Patient_Bolus_Volume_Left - 100;
                    if Patient_Bolus_Volume_Left = 0 then
                        Clinician_Bolus_Paused_Temp := Clinician_Bolus_Paused.Get;
                        if Clinician_Bolus_Paused_Temp then
                            State.Put(Pca_Types.Square_Bolus);
                            Clinician_Bolus_Paused.Put(False);
                        else
                            State.Put(Pca_Types.Basal);
                        end if;
                    end if;
                end if;
            when Pca_Types.Square_Bolus =>
                -- basal
                Flow_Rate := Prescription.Get_Basal_Flow_Rate;
                Period := Get_Time_Between_Activations(Flow_Rate);
                if Last_Basal_Pulse + Ada.Real_Time.Milliseconds(Period) <= Now then
                    Last_Basal_Pulse := Now;
                    Fluid_Pulses.Inc(Integer_Array_Index'Last); -- each time round, update the volume infused
                    Pca_Engine.Run_Pumping(100); -- and pump 0.1 ml
                end if;

                -- clinician
                Clinician_Bolus_Duration_Temp := Clinician_Bolus_Duration.Get;
                Clinicaian_Bolus_Vtbi := Prescription.Get_Vtbi;
                Period := Get_Time_Between_Activations(Pca_Types.Flow_Rate(Natural(Clinicaian_Bolus_Vtbi) * (60/Natural(Clinician_Bolus_Duration_Temp))));
                if Last_Clinician_Bolus_Pulse + Ada.Real_Time.Milliseconds(Period) <= Now then
                    Last_Clinician_Bolus_Pulse := Now;

                    if Clinician_Bolus_Volume_Left = 0 then
                        Clinicaian_Bolus_Vtbi := Prescription.Get_Vtbi;
                        Clinician_Bolus_Volume_Left := Natural(Clinicaian_Bolus_Vtbi) * 1000;  -- in microliters
                    end if;

                    Fluid_Pulses.Inc(Integer_Array_Index'Last); -- each time round, update the volume infused
                    Pca_Engine.Run_Pumping(100); -- and pump 0.1 ml

                    Clinician_Bolus_Volume_Left := Clinician_Bolus_Volume_Left - 100;
                    if Clinician_Bolus_Volume_Left = 0 then
                        State.Put(Pca_Types.Basal);
                    end if;
                end if;
            end case;
        end loop;
    end Rate_Controller;

    task body Max_Drug_Per_Hour_Watcher is
        Release_Time : Ada.Real_Time.Time;
        Period : constant Integer := 1000; -- update in every second
        Volume_Infused : Integer;
        Max_Drug_Per_Hour : Pca_Types.Drug_Volume;
    begin
        loop
            --# assert true;

            Release_Time := Ada.Real_Time.Clock; -- must be simple assignment
            Release_Time := Release_Time + Ada.Real_Time.Milliseconds(Period);
            delay until Release_Time;
            Fluid_Pulses.Pulse; -- each time round, update the volume infused moving window
            Max_Drug_Per_Hour := Prescription.Get_Max_Drug_Per_Hour;
            Volume_Infused := Get_Volume_Infused;
            if Volume_Infused > (Integer(Max_Drug_Per_Hour)*1000) then -- convert to microliters
                State.Put(Pca_Types.KVO);
            end if;
        end loop;
    end Max_Drug_Per_Hour_Watcher;
end Pca_Operation;
\end{lstlisting} 
\doublespacing


\singlespacing
\begin{lstlisting}[language=ada, gobble=0, numbers=left, caption={\lstinline{Pca_Engine} package}, label={listing:pca_ravenscar:pca_engine}]
with Ada.Real_Time;
use type Ada.Real_Time.Time;
--# inherit Ada.Real_Time;
package Pca_Engine
is
    procedure Start_Pumping;

    procedure Stop_Pumping;

    procedure Run_Pumping(Microliters : in Natural);
    --# global in Ada.Real_Time.ClockTime;
    --# derives null from Microliters, Ada.Real_Time.ClockTime;
    --# pre Microliters > 0;

end Pca_Engine;

with Ada.Strings.Unbounded;
use Ada.Strings.Unbounded;
with Ada.Text_IO.Unbounded_IO;
use Ada.Text_IO;

package body Pca_Engine
--# hide Pca_Engine
is
    GPIO_Path : constant String := "/sys/class/gpio/";
    Status_File_Path : constant String := "/home/root/pump_status.txt";

    GPIO_Export_File_Path : constant String := GPIO_Path & "export";
    GPIO_Unexport_File_Path : constant String := GPIO_Path & "unexport";
    GPIO162_Direction_File_Path : constant String := GPIO_Path & "gpio162/direction";
    GPIO162_Value_File_Path : constant String := GPIO_Path & "gpio162/value";

    procedure Start_Pumping
    is
        F    : Ada.Text_IO.File_Type;
        Data : Unbounded_String := To_Unbounded_String("pumping");
        File_Export : Ada.Text_IO.File_Type;
        File_Direction : Ada.Text_IO.File_Type;
    begin
        Create(File_Export, Ada.Text_IO.Out_File, GPIO_Export_File_Path);
        Put_Line(File_Export, "162");
        Close(File_Export);

        Create(File_Direction, Ada.Text_IO.Out_File, GPIO162_Direction_File_Path);
        Put_Line(File_Direction, "out");
        Close(File_Direction);

        Create(F, Ada.Text_IO.Out_File, Status_File_Path);
        Unbounded_IO.Put_Line(F, Data);
        Put_Line("Pumping...");
        Close(F);
    end Start_Pumping;


    procedure Stop_Pumping is
        F    : Ada.Text_IO.File_Type;
        Data : Unbounded_String := To_Unbounded_String("IDLE");
        File_Unexport : Ada.Text_IO.File_Type;
    begin
        Create(File_Unexport, Ada.Text_IO.Out_File, GPIO_Unexport_File_Path);
        Put_Line(File_Unexport, "162");
        Close(File_Unexport);

        Create(F, Ada.Text_IO.Out_File, Status_File_Path);
        Unbounded_IO.Put_Line(F, Data);
        Put_Line("Idle...");
        Close(F);
    end Stop_Pumping;

    procedure Write_Signal(Signal : in Integer) is
        File : Ada.Text_IO.File_Type;
        Data : Unbounded_String;
    begin
        Ada.Text_IO.Open (File => File,
                          Mode => Ada.Text_IO.Out_File,
                          Name => GPIO162_Value_File_Path);
        if Signal = 1 then
            Data := To_Unbounded_String("1");
        else
            Data := To_Unbounded_String("0");
        end if;
        Unbounded_IO.Put_Line(File, Data);
        Ada.Text_IO.Close(File);
    end Write_Signal;

    procedure Run_Pumping(Microliters : in Natural)
    is
        Interval_High: constant Ada.Real_Time.Time_Span := Ada.Real_Time.Microseconds(9000);
        Interval_Low: constant Ada.Real_Time.Time_Span := Ada.Real_Time.Microseconds(1000);
        Next_Time: Ada.Real_Time.Time;
    begin
        Next_Time := Ada.Real_Time.Clock;
        Start_Pumping;
        for I in Integer range 1 .. (15*Microliters) loop
            Next_Time := Next_Time + Interval_High;
            Write_Signal(1);
            delay until Next_Time;
            Next_Time := Next_Time + Interval_Low;
            Write_Signal(0);
            delay until Next_Time;
        end loop;
        Stop_Pumping;
    end Run_Pumping;

end Pca_Engine;
\end{lstlisting} 
\doublespacing

\singlespacing
\begin{lstlisting}[language=ada, gobble=0, numbers=left, caption={\lstinline{Main} procedure}, label={listing:pca_ravenscar:main}]
with Pca_Operation;
with Pca_Types;
with Ice_Types;
with Ada.Text_IO;
with Ada.Float_Text_IO;
--# inherit Pca_Operation,
--#         Ada.Real_Time,
--#         Pca_Types,
--#         Ice_Types;
--# main_program;
procedure Main
is
    pragma Priority (10);
    Input : String(1..10) := (others => ' ');
    Input_Last : Integer;
    Option : Integer;
    use Ada.Text_IO;
begin

    Put_Line("Menu: ");
    Put_Line("0 - Enter prescription");
    Put_Line("1 - Start PCA Pump");
    Put_Line("2 - Stop PCA Pump");
    Put_Line("3 - Display Volume Infused");
    Put_Line("4 - Display Prescription");
    Put_Line("5 - Set Basal Flow Rate");
    Put_Line("6 - Patient bolus");
    Put_Line("7 - Clinician bolus");
    Put_Line("8 - Display Current State");
    loop
        Input := (others => ' ');
        Get_Line(Input, Input_Last);
        Option := Integer'Value(Input);
        case Option is
            when 0 =>
                Put_Line("Enter Basal Flow Rate (ml/hr): ");
                Input := (others => ' ');
                Get_Line(Input, Input_Last);
                Pca_Operation.Panel_Set_Basal_Flow_Rate(Pca_Types.Flow_Rate(Integer'Value(Input)));
                Put_Line("Enter Volume to be infused during patient bolus (ml): ");
                Input := (others => ' ');
                Get_Line(Input, Input_Last);
                Pca_Operation.Panel_Set_Vtbi(Pca_Types.Drug_Volume(Integer'Value(Input)));
                Put_Line("Enter Max Drug Per Hour (ml): ");
                Input := (others => ' ');
                Get_Line(Input, Input_Last);
                Pca_Operation.Panel_Set_Max_Drug_Per_Hour(Pca_Types.Drug_Volume(Integer'Value(Input)));
                Put_Line("Enter Minimum Time Between Boluses (min.): ");
                Input := (others => ' ');
                Get_Line(Input, Input_Last);
                Pca_Operation.Panel_Set_Minimum_Time_Between_Bolus(Ice_Types.Minute(Integer'Value(Input)));
            when 1 =>
                Pca_Operation.StartPump;
                Put_Line("Pump Started");
            when 2 =>
                Pca_Operation.StopPump;
                Put_Line("Pump Stopped");
            when 3 =>
                Put("Volume Infused (ml):");
                Ada.Float_Text_IO.Put(Float(Pca_Operation.Get_Volume_Infused) / Float(1000), AFT=>2, EXP=>0);
                Put_Line("");
            when 4 =>
                Put_Line("Current Basal Flow Rate (ml/hr): " & Integer'Image(Integer(Pca_Operation.Panel_Get_Basal_Flow_Rate)));
                Put_Line("Current Volume to be Infused (ml): " & Integer'Image(Integer(Pca_Operation.Panel_Get_Vtbi)));
                Put_Line("Current Max Drug Per Hour (ml): " & Integer'Image(Integer(Pca_Operation.Panel_Get_Max_Drug_Per_Hour)));
                Put_Line("Current minimum time between bolus (min): " & Integer'Image(Integer(Pca_Operation.Panel_Get_Minimum_Time_Between_Bolus)));
            when 5 =>
                Put_Line("Enter Basal Flow Rate (ml/hr): ");
                Input := (others => ' ');
                Get_Line(Input, Input_Last);
                Pca_Operation.Panel_Set_Basal_Flow_Rate(Pca_Types.Flow_Rate(Integer'Value(Input)));
            when 6 =>
                Pca_Operation.PatientBolus;
            when 7 =>
                Put_Line("Enter Duration (min): ");
                Input := (others => ' ');
                Get_Line(Input, Input_Last);
                Pca_Operation.ClinicianBolus(Ice_Types.Minute'Value(Input));
            when 8 =>
                case Pca_Operation.Get_State is
                    when Pca_Types.Stopped =>
                        Put_Line("Stopped");
                    when Pca_Types.Bolus =>
                        Put_Line("Bolus");
                    when Pca_Types.Basal =>
                        Put_Line("Basal");
                    when Pca_Types.KVO =>
                        Put_Line("KVO");
                    when Pca_Types.Square_Bolus =>
                        Put_Line("Square Bolus");
                end case;
            when others => exit;
        end case;
    end loop;
    Put_Line("Program end");
end Main;
\end{lstlisting} 
\doublespacing