%!TEX root = etdrtemplate.tex

\cleardoublepage


\chapter{PCA Pump Prototype Implementation}
\label{pcapumpimpl}

Currently SPARK 2014 does not support tasking \cite{Spark2014refManual:Online}. For SPARK 2005, GNAT compiler provides Ravenscar Profile \cite{Ravenscar:Online}. It provides a subset of the tasking facilities of Ada95 and Ada 2005 suitable for the construction of high-integrity concurrent programs.

In real-world applications, the embedded critical components are written in SPARK while the non-critical components are written in Ada. Components written in Ada should be hidden for SPARK Examiner with \lstinline{--# hide} annotation.

The biggest challenge during PCA Pump development was the SPARK limitations. There are many common libraries, which cannot be verified by SPARK tools. Thus it required to isolate some functionalities or implement them in different way. An example might be reading and writing numbers to standard input.


\section{Concurrency in SPARK Ada}
\label{pcapump:implementation:concurrency}

Based on AADL models, PCA Pump has to be multitasking device. Thus, concurrency features are needed. In SPARK 2005, concurrency is enable with Ravenscar profile \cite{Ravenscar:Online}. For now, concurrency is not allowed in SPARK 2014.





\section{Implementation based on Requirements Document}

The first step, was to check if implementation of PCA Pump specified in Requirements Document is possible. To do that, simple version of PCA Pump based on Requirements Document was created. Only two AADL threads are implemented: \lstinline{Rate_Controler} and \lstinline{Max_Drug_Per_Hour_Watcher}.



\section{Code generation from AADL models}

Skeleton code generated from AADL models. Then implemented.

Show generated code.



\section{Implementation for generated code}

Overview of issues solved: 
* Bolus options: FBasal + FPatient or FPatient => implemented: FBasal + FPatient (consistent in doc)
5 modes:
* Stopped: F=0
* KVO: F=0.1
* Basal: F=Fbasal
* Patient: F = Fbasal + Fbolus (for vtbi/Fbolus)
* Clinician: F = Fbalsal + Fbolus (for specified time)

Most common Examiner\cite{Examiner:Online} erroes/warnings:
***        Warning                     :302: This expression may be
***        Semantic Error              :725: Protected function or variable XXX may only appear directly in an assignment or return statement.

Discuss implementation of basal infusion: 0.1 ml pulses timed according to the desired rate. (based on CADD-Prizm page 14). Easier bolus monitoring/calculations. Possibility to separate pulse from engine logic. Just array with time stamps(?) or array with size = (60 * 60 /min\_possible\_time\_between\_activations) and set 1 if activation occured. In every second, update array: array[i]=array[i+1]. Array is protected object, so bolus thread cannot access it in the same time, when update thread.
Another option: constant speed of engine and speed-up on boluses. Harder bolus monitoring/calculations?


Internal calculations are in micro liters 1 micro liters ($\mu$l) = 0.001 ml thus 1 ml = 1000$\mu$l.


