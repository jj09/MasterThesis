%!TEX root = etdrtemplate.tex

\cleardoublepage


\chapter{PCA Pump Prototype Implementation}
\label{pcapumpimpl}


Currently SPARK 2014 does not support tasking \cite{Spark2014refManual:Online}. For SPARK 2005, GNAT compiler provides Ravenscar Profile \cite{Ravenscar:Online}. It provides a subset of the tasking facilities of Ada95 and Ada 2005 suitable for the construction of high-integrity concurrent programs.

In real-world applications, the embedded critical components are written in SPARK while the non-critical components are written in Ada. Components written in Ada should be hidden for SPARK Examiner with \lstinline{--# hide} annotation.

The biggest challenge during PCA Pump development was the SPARK limitations. There are many common libraries, which cannot be verified by SPARK tools. Thus it required to isolate some functionalities or implement them in different way. An example might be reading and writing numbers to standard input.


\section{Concurrency in SPARK Ada}
\label{pcapump:implementation:concurrency}

Based on AADL models, PCA Pump has to be multitasking device. Thus, concurrency features are needed. In SPARK 2005, concurrency is enable with Ravenscar profile \cite{Ravenscar:Online}. For now, concurrency is not allowed in SPARK 2014.



\section{BeagleBoard-XM}
\label{pcapumpimpl:beagleboard}


Expansion port 14(PWM) and 28(GND?)
GPIO158
Java Program to Run the pump for 10 seconds

Include source of simple program?
GNAT cross-compiler only for Linux Platform (cross-compilation has to be done on Linux).

compilation+linking command: \lstinline{arm-linux-gnueabi-gnatmake -d -Ppca_ravenscar.gpr}.

%describe how to run the program: compile cmd with arm-gnuaebi... then copy to the board etc.

Running programs on BB (ch.3 from 721 paper)

To run SPARK Ada programs on BeagleBoard-xM GNAT cross-compiler was used. It compiles and links SPARK Ada 2005 single and multitasking programs without any problems. In case of SPARK Ada 2014 only single tasking programs are allowed. Tasking features are not yet in SPARK 2014, which is under development. After successful compilation and linking, there was no issues with running programs on BeagleBoard-xM. Moreover, their behaviour on Intel processor (PC or MacBook) is the same like on ARM device.


\section{Implementation based on Requirements Document}

In this thesis, only the operation module is implemented. Req doc don't specified all implementation details, like how basal rate should be infused.

The first step, was to check if implementation of PCA Pump specified in Requirements Document is possible. Especially whether it will work on BeagleBoard-xM device. To do that, simple version of PCA Pump based on Requirements Document was created. Only two AADL threads are implemented: \lstinline{Rate_Controler} and \lstinline{Max_Drug_Per_Hour_Watcher}.

Pump internal implementation based on \cite{CADD-PrizmAmbulatoryInfusionPump:Online}.
- basal dose deliver in increments - easier to track delivered amount (page 14)

The pump has three tasks in total:
\begin{itemize}
    \item main task (requesting boluses, displaying volume infused etc.)
    \item rate controller - to control the speed of infusion rate
    \item max drug per hour watcher - to control overinfusion
\end{itemize}

The main PCA Pump Package can be verified using SPARK tools. The Pump engine (motor) is separated entity. It has an interface allowing request infusion of 0.1 ml of drug.

PCA Pump prototype was developed based on Requirements Documentation created by Brian Larson, John Hatcliff and Partice Chalin \cite{PcaReq}.


\section{Code generation from AADL models}

PCA Pump prototype was mocked using translations schemas from chapter \ref{codegen}.

The original models were simplified and truncated for the purpose of this thesis. Finally only \lstinline{PCA_Operation} module with 3 threads (\lstinline{Max_Drug_Per_Hour_Watcher}, \lstinline{Rate_Controller}, \lstinline{Patient_Bolus_Checker}), types definitions (\lstinline{Base_Types}, \lstinline{PCA_Types}, \lstinline{ICE_Types}, \lstinline{Bless_Types}) and property set \lstinline{PCA_Properties} were used as the source for code translation.

[code listings of aadl model]


Skeleton code 'generated' from simplified AADL models. Then implemented.

Show generated code.



\section{Implementation for generated code}

Mocked code was extended (e.g. to Prescription\_Store Set/Get methods for record fields were added).

Overview of issues solved: 
* Bolus options: FBasal + FPatient or FPatient => implemented: FBasal + FPatient (consistent in doc)
5 modes:
* Stopped: F=0
* KVO: F=0.1
* Basal: F=Fbasal
* Patient: F = Fbasal + Fbolus (for vtbi/Fbolus)
* Clinician: F = Fbalsal + Fbolus (for specified time)

Most common Examiner\cite{Examiner:Online} erroes/warnings:
***        Warning                     :302: This expression may be
***        Semantic Error              :725: Protected function or variable XXX may only appear directly in an assignment or return statement.

Discuss implementation of basal infusion: 0.1 ml pulses timed according to the desired rate. (based on CADD-Prizm page 14). Easier bolus monitoring/calculations. Possibility to separate pulse from engine logic. Just array with time stamps(?) or array with size = (60 * 60 /min\_possible\_time\_between\_activations) and set 1 if activation occured. In every second, update array: array[i]=array[i+1]. Array is protected object, so bolus thread cannot access it in the same time, when update thread.
Another option: constant speed of engine and speed-up on boluses. Harder bolus monitoring/calculations?


Internal calculations are in micro liters 1 micro liters ($\mu$l) = 0.001 ml thus 1 ml = 1000$\mu$l.


