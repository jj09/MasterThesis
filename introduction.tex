%!TEX root = JakubJedryszek-MasterThesis.tex

\cleardoublepage

\chapter{Introduction}
\label{introduction}

Software is present in all aspects of our lives, from the simple program in alarm clocks to iPads, through cars, refrigerators and computers. Furthermore, our lives are getting more and more dependent on software. Usually when we think about software, we think about applications for PC or smart phone, e.g. calculator, word processor or stock market application. In this case, rapid development and smooth operation is a key. However, there is also another, very important class of software: safety-critical systems. It is comprised of software for airplanes, medical devices, satellites, and rockets.

Software Engineering for real-time safety-critical systems is very different than creating business applications. In both types of software we want to ensure correctness and security. However, in each of them, to a different extent. In the case of the aforementioned word processor, software assurance is not critical. When it crashes, it can be restarted. In worst case scenario, some work might be lost. Airplane software errors may put human lives in danger or even cause death. Thus for safety-critical systems, the security and correctness are crucial. Behind these reasons, different software design methodology, different properties of programming languages and verification tools are needed.

[only about verification]
Part of safety-critical systems design is hazard analysis (identification of different type of hazards), how to avoid unintentional states and how to recover from them. Hazards can cause incidents or accidents. The former is an event that does not cause a loss but is undesired and could lead to an accident. The latter causes the loss, and it is also undesired. Hazard analysis can be done manually by humans or automatically by software tools. Software verification can detect hazards by discovering all possible program states and execution paths.


\section{Motivation}
\label{introduction:motivation}

Nowadays, medical devices work rather independently. This leads to many accidents, which could have been avoided by their interoperability. For example, over-dose of a drug (e.g. morphine) delivered by the patient-controlled analgesia (PCA) pump after surgery can lead to low blood oxygenation or even lack of pulse [CITE FROM dr. Hatcliff]. That can lead to patient's death. The PCA pump does not monitor an oxygen level, but oxygen monitoring device does. If these two devices are organized in centralized system, which implements safety interlock mechanism to shutdown the pump when low blood oxygenation\footnote{Blood oxygenation is also referred as SpO_2} is detected, accident can be avoided. 

In order to communicate, devices have to use compatible interfaces and protocols. There is a concept of "Integrated Clinical Environment" (ICE). It is a standard ASTM F2761, which describes a functional architecture for inter operable systems [CITE FROM DR.HATCLIFF]. The "Laboratory for Specification, Analysis, and Transformation of Software" (SAnToS) created "Medical Device Coordination Framework" (MDCF), which is prototype implementation of ICE. The MDCF vision for ICE is to have requirements documents and conforming software and hardware models. This will enable different medical devices, created by different vendors, to be connected and work under supervision of a centralized system.

In last decades, model-driven development [CITE BOOK FROM AMAZON] became standard for safety critical systems design. [why it is good?] The model-driven development approach proposed in this thesis is a response for the need to create code from models. The PCA pump prototype created in this thesis is as an example of a medical device, which ultimately will work under MDCF.


\section{Technologies}
\label{introduction:technologies}

AADL (Architecture Analysis \& Design Language) \cite{AadlBook} is a modeling language for representing hardware and software. It is used for real-time, safety critical and embedded systems [https://wiki.sei.cmu.edu/aadl/index.php/2010_Publications#ERTS_Feiler_2010 http://web1.see.asso.fr/erts2010/Site/0ANDGY78/Fichier/PAPIERS%20ERTS%202010%202/ERTS2010_0105_final.pdf ]. AADL allows for the description of both software and hardware parts of a system. It is used to describe architecture, but AADL allows to add behavioral extensions through annex languages. BLESS (Behavior Language for Embedded Systems with Software) \cite{Bless:Paper} is an AADL annex sub language defining behavior of components. The goal of BLESS is to automatically check the correctness of AADL models.

Ada is one of the most popular programming languages (along with C/C++) targeted at embedded and real-time systems. SPARK Ada \cite{Barnes:Book} is a subset of Ada, designed for the development of safety and security critical systems. This subset is designed to facilitate static analysis and program verification, which allows to reason about and prove correctness of programs and their entities. There are also SPARK tools for software verification, including tools provided by Altran UK and AdaCore (the developers of SPARK) as well as research groups such as SAnToS Laboratory at Kansas State University.


\section{Contribution}
\label{introduction:contribution}
This thesis demonstrates mapping of AADL/BLESS models to code in SPARK Ada. Additionally it presents current possibilities and limitations of SPARK Ada language, Ravenscar profile and SPARK verification tools. The main contributions of this thesis are as follows:
\begin{itemize}
	\item Review of "Open Patient-Controlled Analgesia Infusion Pump System Requirements" \cite{PcaReq,OpenSourcePCAPump:Paper}.
	\item Identification and analysis of PCA pump and Infusion pumps properties and internals required for implementation.
	\item Cross-compilation and testing of SPARK Ada 2005 and 2014 programs on BeagleBoard-xM platform.
	\item Implementation of PCA pump based on \cite{PcaReq} and AADL/BLESS models.
	\item AADL/BLESS to SPARK Ada translation schemes.
	\item Translation of simplified PCA Pump models (based on created translation schemes).
	\item Design requirements for AADL/BLESS to SPARK Ada translator.
	\item Practical demonstration of SPARK 2005 and SPARK 2014 verification tools: its capabilities and limitations:
		\begin{itemize}
			\item SPARK Examiner
			\item SPARKSimp
			\item Proof Obligation Summarizer (POGS)
			\item Bakar Kiasan
			\item GNATprove
		\end{itemize}
\end{itemize}


\section{Organization}
\label{introduction:organization}
This thesis is organized as follows:
\begin{itemize}
	\item Chapter \ref{background} is background that gives details about ICE, MDCF, Model-Driven Development, AADL, BLESS, SPARK Ada and its verification tools. 
	\item Chapter \ref{pcapump} describes Patient-Controlled Analgesia (PCA) pump.
	\item Chapter \ref{codegen} presents mappings from AADL/BLESS to SPARK Ada. 
	\item Chapter \ref{pcapumpimpl} describes the implementation of PCA Pump Prototype. Faced issues and design decisions made.
	\item Chapter \ref{verification} describes verification of implemented PCA Pump Prototype and code translated from simplified version of AADL models. 
	\item Chapter \ref{summary} summarizes all work which has been done in this thesis. 
	\item Chapter \ref{future_work} is the future work that can be done in this topic.
\end{itemize}
